\section{Published UV RRS studies of NA}

First UV RRS experiments concerned mainly the simple NA components.
The published works from the seventies and eighties were devoted primarily to
nucleotides, probably because of their good solubility
\parencite{Tsuboi1974}.
UV RRS investigations of nucleotides continued in the eighties, usually with
the aim to specify precisely the dependence of the resonance enhancement on the
excitation wavelength
\parencite{Ziegler1984}.
From the half of eighties, relatively precise UV RRS measurements of
nucleobases and their analogs appeared, usually with the aim to enlarge the
knowledge of the nucleobase vibrational spectra to support the interpretation
of vibrational lines
\parencite{Ghomi1986}.
Other studies focused on the effect of nucleobase protonation state and/or
hydrogen bonds (both donor/acceptor type) on the UV RRS spectra of nucleotides
\parencite{Gfrorer1991}.

Polynucleotides have been popular molecular models of nucleic acids for their
accessibility and the limited number of different nucleobases, which simplified
the assignment of Raman lines.
They can simulate various forms of folded NA structures like a double
helix of A, B, or Z conformation, triplex, or ordered single-helical structure.
Early UV RRS studies were focused on the effect of resonance Raman
hypochromism
\parencite{Pezolet1975}
and hyperchromism
\parencite{Chinsky1980}.
First measurements of complete temperature spectral dependences, i.e., UV RRS
melting curves of NA, were carried out on a set of polynucleotides forming A/T
or A/U base pairs
\parencite{Jolles1985}.

Changes of Raman line positions caused by the duplex formation were also
measured
\parencite{Grygon1990}.
RRS conformational markers of the triple helix
poly(U)$\cdot$poly(A)$\cdot$poly(U), the double helix poly(A)$\cdot$poly(U),
and a random copolymer poly(AU) (260 and 220\,nm excitation) were studied from
neutral to low pH (down to 2.5)
\parencite{Gfrorer1993a}.
\textcite{Wheeler1996}
demonstrated that quadruplex structures are present at
high ionic strength.

The published UV RRS measurements of natural NA are relatively rare
\parencite{%
	Laigle1982a,%
	Fodor1986a,%
	Wen1999,%
	Neugebauer2007,%
	Shaw2009%
}.
Oligonucleotides as NA segments with custom specified nucleobase sequences
can serve as realistic molecular models of important NA structural elements.
The published UV RRS studies concern mainly duplex structures
\parencite{Laigle1986},
hairpins
\parencite{Refregiers1997},
and guanine quadruplexes
\parencite{Mukerji1995}.

In addition to the studies of nucleic acids themselves and their components,
several works were published on their interactions with other molecules.
Numerous works concerned assorted biologically significant molecules, which
interacted with NA preferentially via intercalation between stacked
nucleobases \parencite{Chinsky1978}.
A few UV RRS works were devoted to NA interaction with peptides or protein
segments \parencite{Laigle1982a}.
A particular group of works is the studies of nucleic acid interactions with
cis-platinum complexes
\parencite{Perno1987}.
