\section{The most important results}

UV RRS measurements of NA components providing first data concerning the
\emph{resonance Raman enhancement} (RRE) of nucleobase vibrational modes and
its dependence on the excitation wavelength stimulated intensive efforts for
RRE theoretical prediction.
It was assumed that the comparison with experimental results would determine
the equilibrium value shift of the vibrational coordinates
connected with a particular electronic excitation
\textcite{Peticolas1970}.
Further works revealed the need to consider also the electron transitions lying
at higher energies
\parencite{Chinsky1980}.
The UV RRS intensities were used to improve the molecular force fields for
nucleic acid bases
\parencite{Lagant1991}.

Obviously, the most precise theoretical calculations of UV RRS spectra of NA
components published so far are the works of
\textcite{%
	Sun2014,%
	Sun2015,%
	Sun2017%
}.
They used the Herzberg-Teller short-time dynamics formalism considering three
electron transitions --
	the low-energy \g{p}-\g{p}* transition,
	the low-energy n-\g{p}* transition,
	and Rydberg transition at higher energy
-- and made their calculation for nucleobase complexes with explicit water
molecules.
The authors revealed a strong effect of hydrogen bonds on the vibrational
frequencies and RRS intensities as well.
The optimal inclusion of explicit hydrogen bonding might be the way to match
the experiment better.

In parallel to the first direct UV RRS calculations, another idea of analyzing
UV RRS spectra has appeared.
It was based on a match between the theoretical expressions for UV RRS and
electron absorption.
This enabled the construction of semiempirical approaches when the shape of the
electronic absorption spectrum was employed to substitute a significant part of
the UV RRS calculation
\parencite{Blazej1980}.
A certain similarity to this approach is the method elaborated by the Loppnow
group, which is based on the similarity of the expressions for the
time-developed perturbation before the integration by time.
This analysis, called by the authors “time-dependent wave packet formalism” was
applied to UV RRS of numerous NA components and their derivatives
\parencite{Billinghurst2006a}.
Despite the obviously higher quality of UV RRS spectra measured using
up-to-date equipment, the agreement between the predicted and measured
excitation profiles was not always satisfying.

The resonance enhancement concerns almost exclusively in-plane vibrational
modes of nucleobases.
UV RRS can thus serve as an elementary verification of the proposed
interpretation of vibrational transitions observed in non-resonant RS or IR
absorption spectra.
On the other hand, frequencies of some NA lines active in RRS are sensitive to
conformational changes, which indicate their coupling to the sugar vibrational
modes
\parencite{Nishimura1987}.

The obtained RRE can also be employed to confirm the vibrational mode
interpretation provided the characteristics of the respective electron
transition are known or at least realistically estimated
\parencite{Fodor1985}.

A significant advantage of the resonance RS over the classical one is that it
is relatively simple (lower number of active fundamental transitions).
Moreover, the use of several excitation wavelengths allows for the separation
of overlapping bands.
This makes it possible to accurately monitor the frequency changes caused by
various molecular states or identify individual bands in complex NA samples
containing all types of nucleobases
\parencite{Mukerji1995}.
UV RRS was also successfully employed to analyze premelting conformational
changes in poly(dA)$\cdot$poly(dT) and poly(dAdT)
\parencite{Chan1997}.
The suitability of UV RRS to study complex NA samples was demonstrated by the
study of tandem repeats of telomeric DNA, when numerous markers of guanine
quadruplexes were established
\parencite{Krafft2002},
or by monitoring A to B transition in DNA dodecamers
\parencite{Knee2008}.
