\chapter{UV RRS measurements of nucleic acids}

Despite the decades of intensive research, \emph{nucleic acids} (NA) represent
still a permanent object of the structural studies, which focus on
characterizing NA local geometric arrangement, its formation, and collapse
depending on the internal (nucleobase sequence) and external (environmental
conditions, intermolecular interactions) parameters.
The reason is the rich structural polymorphism of nucleic acids, which are in
addition to the classic Watson-Crick duplex capable of adopting various
non-canonical structures, such as hairpins, cruciforms, triplexes,
quadruplexes, or i-motifs.
It has also been shown that the processes of the expression of genetic
information, of its modification and transfer are controlled through specific
nucleic acid-protein interactions.
However, detailed molecular mechanisms of these fundamental biological
processes are often unknown
\parencite{%
	Bansal2014,%
	Kaushik2016,%
	Yamamoto2021%
}.

A variety of experimental techniques has been used to investigate the
structural arrangement of nucleic acids in their natural environment.
Many of them were gradually developed to achieve credible and
well-interpretable results.
One of the relatively new techniques considered as perspective is the
resonance Raman scattering excited in the ultraviolet spectral region (UV RRS).
UV RRS of NA and their components have been measured since the beginning of
the seventies.
In contrast to the original optimistic expectations, this method is though
currently applied in nucleic acid studies only rarely.

Within the fifty years from the first published works dealing with UV RRS
measurements of NA, the devices used have changed according to the technical
progress and the growth experience with this type of experiment.
We concentrate on the main parts of UV RRS spectrometers, including the way of
the sample placement.

It is typical for the first two decades that a significant number of various
laser sources were employed to reach an intensive excitation at various UV
wavelengths.
As a rule, pulse lasers with high peak power were chosen to reach the high
efficiency of nonlinear effects serving for the wavelength transformation.
At the beginning of the nineties,
the cw Ar\textsuperscript{+} ion laser with intracavity
doubling has appeared as a new UV source providing 257, 248, 244, 238 and
228.9\,nm excitations
\parencite{%
	Asher1993b%
}.
Due to the cw regime, its use lowered nonlinear sample photochemistry, sample
thermal degradation, and ground-state saturation.

Both 90-degree and backscattering geometries were employed.
More diverse were the arrangements of the collecting paths.
Besides the quartz lenses, mirrors (concave mirrors
\cite{Blazej1977},
or later Cassegrain reflective objectives
\cite{%
	Toyama1991%
})
were used to avoid the chromatic aberration.

Resonance Raman spectroscopy uses excitation light with frequency inside the
electronic absorption band of samples.
It means that the investigated molecules accept a significant part of the power
from the incoming laser beam, and this excess energy can destroy the samples.
It also locally increases temperature and causes the thermal lens effect, which
distorts laser focus.
Over time, resonance Raman spectroscopists invented methods to minimize these
effects.

Except for some cases when UV RRS spectra of NA were measured in stationary
quartz capillaries or tubes
\parencite{%
	Blazej1977%
},
three main concepts, quartz rotating cell
\parencite{%
	Kiefer1971,%
	Kiefer1971a,%
},
flow system with a reservoir
\parencite{Ziegler1981}
or a properly designed flow cell
\parencite{Blazej1980},
standard cuvette with constant stirring
\parencite{Jolles1984},
have been designed to reduce the NA photodamage during the
experiment.
All these concepts are used up to now without any precise evaluation, which of
them should be preferred.

During the first period, when the main aim was to determine the UV RRS
excitation profiles of various NA components, great attention was paid to an
intensity calibration.
Surprisingly only a few published UV RRS studies of NA specified how the
wavenumber scale was calibrated.
Usually, the calibrations were based on known positions of Raman lines of
simple compounds.
This issue is discussed in detail in
\cref{wavenumber_calibration}.

First UV RRS experiments concerned mainly the simple NA components.
The published works from the seventies and eighties were devoted primarily to
nucleotides, probably because of their good solubility
\parencite{Tsuboi1974}.

Polynucleotides have been popular molecular models of nucleic acids for their
accessibility and the limited number of different nucleobases, which simplified
the assignment of Raman lines.
They can simulate various forms of folded NA structures like a double
helix of A, B, or Z conformation, triplex, or ordered single-helical structure.
Early UV RRS studies were focused on the effect of resonance Raman
hypochromism
\parencite{Pezolet1975}
and hyperchromism
\parencite{Chinsky1980}.
First measurements of complete temperature spectral dependences, i.e., UV RRS
melting curves of NA, were carried out on a set of polynucleotides forming A/T
or A/U base pairs
\parencite{Jolles1985}.

Changes of Raman line positions caused by the duplex formation were also
measured
\parencite{Grygon1990}.
RRS conformational markers of the triple helix
poly(U)$\cdot$poly(A)$\cdot$poly(U), the double helix poly(A)$\cdot$poly(U),
and a random copolymer poly(AU) (260 and 220\,nm excitation) were studied from
neutral to low pH (down to 2.5)
\parencite{Gfrorer1993a}.
\textcite{Wheeler1996}
demonstrated that quadruplex structures are present at
high ionic strength.

Oligonucleotides as NA segments with custom specified nucleobase sequences
can serve as realistic molecular models of important NA structural elements.
The published UV RRS studies concern mainly duplex structures
\parencite{Laigle1986},
hairpins
\parencite{Refregiers1997},
and guanine quadruplexes
\parencite{Mukerji1995}.

A significant advantage of the resonance RS over the classical one is that it
is relatively simple (lower number of active fundamental transitions).
Moreover, the use of several excitation wavelengths allows for the separation
of overlapping bands.
This makes it possible to accurately monitor the frequency changes caused by
various molecular states or identify individual bands in complex NA samples
containing all types of nucleobases
\parencite{Mukerji1995}.
UV RRS was also successfully employed to analyze premelting conformational
changes in poly(dA)$\cdot$poly(dT) and poly(dAdT)
\parencite{Chan1997}.
The suitability of UV RRS to study complex NA samples was demonstrated by the
study of tandem repeats of telomeric DNA, when numerous markers of guanine
quadruplexes were established
\parencite{Krafft2002},
or by monitoring A to B transition in DNA dodecamers
\parencite{Knee2008}.
