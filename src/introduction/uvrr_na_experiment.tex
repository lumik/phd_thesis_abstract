\section{Experimental Approach}

Within the fifty years from the first published works dealing with UV RRS
measurements of NA, the devices used have changed according to the technical
progress and the growth experience with this type of experiment.
We concentrate on the main parts of UV RRS spectrometers, including the way of
the sample placement.

It is typical for the first two decades that a significant number of various
laser sources were employed to reach an intensive excitation at various UV
wavelengths.
As a rule, pulse lasers with high peak power were chosen to reach the high
efficiency of nonlinear effects serving for the wavelength transformation.

At the beginning of the nineties,
the cw Ar\textsuperscript{+} ion laser with intracavity
doubling has appeared as a new UV source providing 257, 248, 244, 238 and
228.9\,nm excitations
\parencite{%
	Asher1993b,%
	Russell1995%
}.
Due to the cw regime, its use lowered nonlinear sample photochemistry, sample
thermal degradation, and ground-state saturation.
The beam could be focused to a small sample volume, which could be efficiently
collected; this capability resulted in very high spectral SNR.

Both 90-degree and backscattering geometries were employed. An exception is a
grazing incidence geometry used in the case of the sample cuvette with a side
opening
\parencite{Jolles1984}
and a 120\textdegree geometry used for measurement from a jet stream
\parencite{Fodor1985}.
The excitation beam was directed employing planar mirrors and focused by a
quartz lens.
More diverse were the arrangements of the collecting paths.
Besides the quartz lenses, mirrors (concave mirrors
\cite{Blazej1977},
or later Cassegrain reflective objectives
\cite{%
	Toyama1991,%
	Russell1995%
})
were used to avoid the chromatic aberration.

Spectral analysis was first performed using double or triple spectrographs
\parencite{%
	Harada1975,%
	Gfrorer1993a,%
	Toyama1993%
},
from the beginning of the nineties equipped with a cooled CCD detector
\parencite{%
	Gfrorer1993a,%
	Toyama1993%
}.
It was later demonstrated that a single-stage spectrograph (which is of higher
throughput) is sufficient to reject the elastically scattered radiation when it
is placed behind a simple prefiltering element.
The premonochromators, used for this purpose
\parencite{%
	Hashimoto1993,%
	Russell1995%
},
were after replaced by proper optical filters
\parencite{%
	Munro1997,%
	Bykov2013%
}.

Resonance Raman spectroscopy uses excitation light with frequency inside the
electronic absorption band of samples.
It means that the investigated molecules accept a significant part of the power
from the incoming laser beam, and this excess energy can destroy the samples.
It also locally increases temperature and causes the thermal lens effect, which
distorts laser focus.
Over time, resonance Raman spectroscopists invented methods to minimize these
effects.

Except for some cases when UV RRS spectra of NA were measured in stationary
quartz capillaries or tubes
\parencite{%
	Blazej1977,%
	Asher1983%
},
three main concepts, quartz rotating cell
\parencite{%
	Kiefer1971,%
	Kiefer1971a,%
},
flow system with a reservoir
\parencite{Ziegler1981}
or a properly designed flow cell
\parencite{Blazej1980},
standard cuvette with constant stirring
\parencite{Jolles1984},
have been designed to reduce the NA photodamage during the
experiment.
All these concepts are used up to now without any precise evaluation, which of
them should be preferred.

During the first period, when the main aim was to determine the UV RRS
excitation profiles of various NA components, great attention was paid to an
intensity calibration.
Surprisingly only a few published UV RRS studies of NA specified how the
wavenumber scale was calibrated.
Usually, the calibrations were based on known positions of Raman lines of
simple compounds.
This issue is discussed in detail in
\cref{wavenumber_calibration}.
