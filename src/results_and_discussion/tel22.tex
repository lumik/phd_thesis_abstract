\section[%
	Temperature and concentration effects on transition from antiparallel	to
	parallel quadruplex in tel22
]{%
	Temperature and concentration effects on transition from antiparallel to
	parallel\\quadruplex in tel22
}

One class of the unusual nucleic acid conformations is
\emph{guanine quadruplexes} (GQ)
formed by stacked planar guanine tetrads.
The structure is stabilized by Hoogsteen base pairing and cation coordination.
GQ are abundant within human DNA, and their presence in telomeric guanine-rich
regions of human core telomeric repeats \ch{d[TTAG3]_n} is well known.
It was also observed that the GQ exhibit remarkable polymorphism, especially
those stabilized by \ch{K^+} ions
\parencite{Chaires2013}.

It is interesting to investigate GQ in the molecular crowding environment,
physiological temperatures, and in the presence of \ch{K^+} ions which mimicks
their actual \emph{in vivo} conditions in cells.
Recent \emph{in vitro} studies revealed that DNA concentration and thermal
treatment
of the sample might play an important role in GQ folding and interquadruplex
transitions
\parencite{Palacky2013}.

Circular dichroism and conventional Raman spectroscopy revealed that
thermodynamically equilibrated (heated and slowly cooled) samples of Tel22
change from antiparallel to hybrid “3 + 1” and parallel form and that
the fraction of the parallel form increases with the concentration of the
oligonucleotide or \ch{K^+} ions
\parencite{Palacky2013}.

Transition to the parallel quadruplex form was observed with progressing time
(72\,hours in total) if such sample was incubated at a physiological
temperature of 37\,\textdegree{}C.
This temperature is far below the melting point under given
conditions ($> 75$\,\textdegree{}C).
Furthermore, the samples incubated at 37\,\textdegree{}C remained fluid even
after the transition to the parallel form.
A similar spectral change can be seen for the same sample after annealing
(heating to 95\,\textdegree{}C and slowly cooling down to 20\,\textdegree{}C),
but the resulting sample forms a stiff gel.
\Figref{telXXII:spectra} shows the UV RRS results.

\begin{figure}[ht]
	\centering
	\input{results_and_discussion/assets/tel22/tel22_spectra}
	\vspace{3mm}
	\caption[%
		Effect of standard anealing and incubation at 37\,\textdegree{}C on UVRRS
		of \ch{K^+}-Tel22.
	]{%
		\captiontitle{%
			Effect of standard anealing and incubation at 37\,\textdegree{}C on UVRR
			spectra of \ch{K^+}-Tel22.
		}
		The “3 + 1” conformation, which is formed quickly after the addition of
		\ch{K^+} ions to the solution and which has C2'-endo/syn conformation with
		increased intensity of guanine band at 1325\,\icm{}, switches to parallel
		conformation with C2'-endo/anti conformation with increased intensity of
		the guanine band at 1338\,\icm{}.
	}
	\label{\figlabel{telXXII:spectra}}
\end{figure}

These results confirm that UVRRS can be used for studying
different unusual NA conformations such as GQ. It also extends available
concentration ranges and can be used to investigate of the stability of
nucleic acid complexes while keeping structural information contained in the
vibrational spectra.
