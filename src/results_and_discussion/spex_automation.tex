\section{Automation of Raman experiment}

Investigation of the temperature dependence of Raman spectra is an important
technique for studying nucleic acids
\parencite{Klener2021}.
However, high-quality results for multivariate analysis can demand high numbers
of spectra
\parencite{Palacky2020}.
Obtaining large sets of temperature-dependent spectra would be nearly
impossible without appropriate automation.
Automation also reduces space for human errors and improves precision in
timing.

Our laboratory is equipped with a homemade Raman spectrometer with visible
laser excitation suitable for precise temperature-dependent measurements.
The spectrometer uses 90\textdegree{} scattering geometry, Jobin-Yvon Spex 270M
monochromator, and a liquid nitrogen-cooled CCD detector from Princeton
Instruments.
The precise wavenumber calibration utilizes a neon glow lamp, and the
calibration spectrum is taken after each spectrum to compensate for all
time-based drift in the apparatus.
This approach allows for calibration of the wavenumber scale with precision in
the range of 0.1\,\icm{}.

The temperature of the samples is controlled by a water bath (Neslab).
The original
temperature-dependent Raman measurement with this setup meant that the
scientist needed to manually change temperature and switch between calibration
and sample measurement modes which required almost continuous presence
in the laboratory.

Some experiments performed during this work took advantage of complementary
measurements of the temperature dependence of visible and UV Raman spectra
\parencite{Klener2021},
and therefore it was decided to automate the measurements with varying
temperatures on the visible Raman spectrometer.
This automation, carried out within the framework of the dissertation, also
enabled other scientific works in our laboratory
\parencite{%
	Mudronova2016,%
	Bravo2018,%
	Palacky2020%
}.

The newly created experiment automation program \emph{SpExpert} can control the
CCD detector, spectrograph, switch between calibration and sample measurement
position, switch of calibration, lamp and temperature of the thermostated
water bath.
The whole SpExpert program is written in C++ using Qt framework
\parencite{Qt}
and is available for download online
\parencite{SpExpert2018}.
It focuses mainly on temperature-based Raman measurements with the possibility
to use extended-range mode which, means that it is possible to measure with
more positions of spectrograph grating for each temperature.
Temperature evolution with time can be reviewed during measurement right
from the program screen, see
\figref{spex_automation:temperature_measurement}.

\begin{figure}
	\centering
	\ig{0.7}{results_and_discussion/assets/spex_automation/%
		spexpert_running_temperature2.jpg}
	\caption[%
			SpExpert program during temperature-dependent measurement.
	]{%
		\captiontitle{%
			SpExpert program during temperature-dependent measurement.
		}%
		The live view of temperature can be seen.}
	\label{\figlabel{spex_automation:temperature_measurement}}
\end{figure}
