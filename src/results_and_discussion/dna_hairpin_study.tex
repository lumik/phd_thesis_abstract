\section{%
	Analysis of temperature effects on the structural arrangement of DNA segments
}

Possibility to obtain reliable UV RRS spectra at various temperatures was
employed in study of a simple DNA structural transition.
Our aim was to demonstrate that UV RRS combines reach informational content of
Raman spectrum with a high sensitivity to weak aspects of temperature induced
changes.
Beside this it was also desired to obtain UV RRS characteristics concerning
particular structural arrangements of DNA hairpin and double helix.

The experiments were carried out on DNA dodeca-deoxynucleotides
d(5'GCCG\allowbreak{}ATTACGGC3'), d(5'GCCGATTAGCCG3'), d(5'CGGCTAATCGGC3').
The first forms a hairpin while the latter two in a 1:1 molar mixture form a
fully complementary duplex.
In both cases, the total base composition was the same, containing (in the
folded state) a double-strand section of GC pairs and a central section of AT
pairs forming a flexible part of the double helix in the duplex or the loop in
the hairpin
(see \figref{dna_hairpins:structure})

\begin{figure}[t]
	\centering
	\ig{1}{results_and_discussion/assets/dna_hairpins/structure}
	\caption[%
		Scheme of an arrangement of the studied dodecadeoxynucleotides in the
		folded state.
	]{%
		\captiontitle{%
			Scheme of an arrangement of the studied dodecadeoxynucleotides in the
			folded state: hairpin (left) and duplex (right).
		}
		Dashes represent the sugar-phosphate links in the chain and dots represent
		the hydrogen bonds connecting the W.-C. basepairs.
	}
	\label{\figlabel{dna_hairpins:structure}}
\end{figure}

To reveal the nature of the spectral changes, the UV RRS spectra and the
spectral series of complementary measurements of non-resonant RS excited and UV
absorption were subjected to Principal Component Analysis (PCA)
\parencite{Malinowski2002}.
For all spectral series, four PCA components were well resolvable while the
fifth and higher components primarily represented the noise contribution.

The scores $P$ were least-square fitted according to a formula considering both
the ratio of the folded and unfolded form governed by the Van’t Hoff equation
and the linear temperature dependence of each form corresponding to the
structural relaxation.
The parameters of the sigmoidal asymptotes obtained by the fits allow us to
isolate spectra of both the folded and unfolded form at any temperature $T$
(assuming linearity of the noncoordinated temperature-induced changes is valid
in the studied temperature range).
The obtained UV RRS spectra and their differences reflecting particular types
of temperature-induced changes are shown in
\figref{dna_hairpins:forms_spectra}.
The detailed spectral analysis was published by
\textcite{Klener2021}.

\begin{figure}[t]
	\centering
	\ig{1}{results_and_discussion/assets/dna_hairpins/forms_spectra}
	\caption[%
		Isolated UV RRS spectra of the folded and unfolded form at several
		temperatures and their differences corresponding to the spectral changes
		caused by the structural transition from the folded to the unfolded form
		(melting) and those caused by a temperature increase when the structural
		form is maintained.
	]{%
		\captiontitle{%
			Isolated UV RRS spectra of the folded and unfolded form at several
			temperatures and their differences corresponding to the spectral changes
			caused by the structural transition from the folded to the unfolded form
			(melting) and those caused by a temperature increase when the structural
			form is maintained.
		}
		The chosen temperature increment of 25\,\textdegree{}C corresponds
		approximately to the width of the temperature interval of the structural
		transition.
	}
	\label{\figlabel{dna_hairpins:forms_spectra}}
\end{figure}

We can conclude that this study demonstrated the ability of UV resonance Raman
spectroscopy to resolve and determine, in contrast to non-resonance RS, all
three types of temperature-induced effects of DNA oligonucleotides involving
more flexible central AT tract.
The reason is in the dominant effect of the oligonucleotide structure
disintegration on the resonance enhancement, which is pronounced via remarkable
changes of the Raman band intensities.
This enables us to distinguish the temperature-induced structural transition
(melting) from the effects of the temperature increase on the oligonucleotide
in both the folded and unfolded forms.

The difference spectra corresponding to melting confirmed similar B-form
double-helical structures of the segments consisting of GC basepairs for both
the duplex and hairpin with a slightly more relaxed geometry in the latter
case.
Minimal stacking has been indicated for the nucleobases in the loop.
Nevertheless, certain interactions of the adenine rings with neighboring
nucleobases have been detected.
Warming of both oligonucleotides in their folded state primarily influences
their segments consisting of GC basepairs.
An unusual adenine position or its unusual interaction with surroundings at low
temperatures has been also indicated.
Warming of the oligonucleotides in their unfolded state seems to demonstrate
the tendency of separated and open oligonucleotide chains to prefer momentary
local geometry with stacked neighboring purine nucleobases.
