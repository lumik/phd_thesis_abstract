\section{Design of UV Raman experiment}

Work on the construction of a new UV Raman spectrometer was an incremental
process.
It was necessary to consider the equipment already present in the lab, mainly
the laser system, CCD camera, and optical table and specify parameters of the
chosen commercial spectrograph.
Then the first functional prototype was constructed and improved for
measurement of polarized UV Raman spectra.
The initial wavenumber calibration technique of measured Raman spectra
was adopted based on the literature.
Then we started to deal with photo-decomposition of samples by designing a
spinning cell that could contain as small as cca. 20\,\g{m}l volumes of
samples.
After that, we widened the range of usable excitation wavelengths of the
instrument and decided to seek a new means of wavenumber calibration, and
after some research and trials and errors, we settled with calibration to the
spectra of Pt lamp.
And then, the apparatus was enhanced to enable temperature measurements in
backscattering geometry.

We used Coherent Innova 300C MotoFreD\texttrademark{} Ion Laser as an
excitation source with BBO crystals (also from Coherent) designed for doubling
514, 488, and 457\,nm resulting in 257, 244, and 229\,nm excitation
wavelengths.
As a detector, we used liquid nitrogen-cooled Princeton Instruments
SPEC-10:2KBUV/LN back-illuminated CCD camera enhanced for UV light detection.
The camera has $2048\times512$ pixels of $13.5\times13.5$\,\g{m}m and can
be controlled with WinSpec software through ST133B/U camera controller unit.
As a spectrograph, we decided on Horiba iHR550 Imaging Spectrometer with
aperture f/6.4, focal length 550 mm, magnification 1.1, and triple-grating
turret equipped with 300\,gr/mm, 1200\,gr/mm, and 2400\,gr/mm gratings.

\label{wavenumber_calibration}
For correct Raman spectra interpretation, we needed precise wavenumber
calibration.
It can be done by measuring spectrum with known line positions in the same
spectrometer configuration as for the spectrum of the sample and then
interpolating and extrapolating the wavenumbers from these known line positions
to all detector pixels.
One group of calibration spectra is Raman spectra of organic solvents, but one
cannot use pure indene, which is widely used in visible Raman because it
strongly absorbs UV light, and its lines can be shifted if it is diluted.
So other solvents need to be found.

We tried calibration to cyclohexane because it has a pretty rich Raman
spectrum.
We also wanted to solve the problem that most standard organic solvents do not
have many bands between 1800 and 2800\,\icm{}, so we decided to use deuterated
cyclohexane-d12, which nicely fills that region.
To our knowledge, the cyclohexane-d12 was not used for calibration of UV Raman
spectra of nucleic acids before.
But, a different method for sub-\icm{} calibration precision
than calibration to organic solvents needs to be used,
so we tried to use a mercury lamp which was reported in some works
\textcite{Manoharan1990},
but it did not bring any new precision to
the calibrated spectra, so we searched for the different calibration sources.

Finally, we found out that the best wavenumber scale calibration results are
obtained with a hollow cathode platinum lamp which broadly used for calibration
of cosmic space instruments
\parencite{%
	Mount1977,%
	Reader1990,%
	Sansonetti1992%
}.
\emph{Hollow cathode lamps} (HCL) are routinely used in \emph{atomic absorption
spectroscopy} (AAS), and therefore they are commercially available on the
market together with power supplies because HCL requires a precision current
source.
We used P209 HCL Power Supply (Photron) with P840 Hollow Cathode Lamp -- Pt
(Photron) which provided a stable platinum atomic spectrum right after start.

We also evaluated crystaline and wire grid polarizers for polarized UV Raman
measurements and we chose Glan-laser polarizer because it was more reliable
and had better throughput.

We created a spinning cell to minimize photodecomposition of samples.
We designed the cell to be also usable with small volumes of the samples, the
inner radius of 4\,mm and height of 5\,mm.
It means that the maximal volume could be $\sim250$\,\g{m}L, but due to
centrifugal force, the smaller amount of samples could be measured;
for example, 50\,\g{m}L of sample results in an $\sim 4$\g{m}m thick layer
attached to the cell wall if the cell is rotated sufficiently fast.
A sample holder was inspired by the work of
\textcite{Shriver1974},
but we used FPM o-rings with an inner diameter of 11\,mm and 1\,mm diameter of
the rubber as displayed in
\figref{spinning_cell:drawing}

\begin{figure}
	\centering
	\begin{tikzpicture}[scale=0.5, font=\sffamily, >=Latex]

\definecolor{glassBorderColor}{RGB}{0,128,255}
\definecolor{glassFillColor}{RGB}{230,242,255}

\tikzset{
	holder cap/.style = {color = black, fill = black!20},%
	holder/.style = {color = black, fill = black!30},%
	holder background/.style = {color = black, fill = black!10},%
	thread/.style = {decoration = {zigzag, segment length = 1.5mm,%
		amplitude = 1mm}},%
	oring/.style = {color = black, fill = black},%
	glass/.style = {draw = none, fill = glassFillColor, fill opacity = 0.5},%
	cell wall/.style = {color = glassBorderColor, opacity = 0.4,%
		fill=glassFillColor, fill opacity = 0.9},%
	cell border/.style = {color = glassBorderColor, opacity = 0.4},%
	stopper/.style = {color = black!30, fill = white, rounded corners = 0.1cm,
		fill = black!10},%
	bar scale/.style = {<->, >=Bar[], line width=1.5*\pgflinewidth},%
	%rotation axis/.style = {color = black!70, line width=1.5*\pgflinewidth,
		%dash pattern = on 5pt off 5pt on 1.5*\pgflinewidth off 5pt}
}

%\clip (-.1,-.1) rectangle (17,20.2);

% cell
\draw [glass] (2.9,0) rectangle ++(11.0,7.0);
% wall
\draw [cell wall] (6.2,7.0) -- ++(-3.3,0) -- ++(0,-7.0) -- ++(11.0,0)
	-- ++(0,7.0) -- ++(-3.3,0) -- ++(0,-1.0) -- ++(1.8,0) -- ++(0,-5.0)
	-- ++(-8.0,0) -- ++(0,5.0) -- ++(1.8,0) -- cycle;
% stopper hole
\draw [cell border] (6.2,7.0) coordinate (lstopperhole)
	-- ++(4.4,0) coordinate (rstopperhole);
\draw [cell border] (6.2,6.0) -- ++(4.4,0);

% stopper
\newcommand{\stopperwidth}{4.8}
\coordinate (stoppercenter) at (8.4,4.0);
\path [name path = topstopper] ($ (stoppercenter)
		+ (-\stopperwidth - 1.0,5.0) $)
	-- ++(2*\stopperwidth + 2.0,0);
% left border
\coordinate (lbstopper) at ($ (stoppercenter) + (-2,0) $);
\path (lbstopper) -- (lstopperhole);
\mypgfextractangle{\lstopperangle}{lbstopper}{lstopperhole}
\path [name path = lstopper] (lbstopper) -- ++(\lstopperangle:6);
\path [name intersections = {of = topstopper and lstopper, by = ltstopper}];
\path [name path = louterstopper] ($ (stoppercenter) + (-3.5,7.0) $)
	-- ++(0,-3.0);
\path [name intersections = {of = topstopper and louterstopper,%
	by = loutertstopper}];
% right border
\coordinate (rbstopper) at ($ (stoppercenter) + (2,0) $);
\path (rbstopper) -- (rstopperhole);
\mypgfextractangle{\rstopperangle}{rbstopper}{rstopperhole}
\path [name path = rstopper] (rbstopper) -- ++(\rstopperangle:6);
\path [name intersections = {of = topstopper and rstopper, by = rtstopper}];
% draw the stopper
\draw [stopper] (lbstopper) -- (ltstopper) -- (loutertstopper)
	-- ++(0,2.0) -- ++(7.0,0) -- ++(0,-2.0) -- (rtstopper) -- (rbstopper)
	-- cycle;

% holder
\draw [holder] (1.3,5.5) -- ++(0.5,0) -- ++(0,0.7) -- ++(1.0,0) -- ++(0,0.8)
	-- ++(1.6,0) -- ++(0,7.0) -- ++(3.0,1.0) -- ++(0,5.0) -- ++(-3.0,0)
	-- ++(0,-5.0) -- ++(-3.1,-3.0) -- ++(-0.4,0) -- ++(0,-4.0) -- ++(0.4,0)
	-- cycle;
\draw [holder] (15.5,5.5) -- ++(-0.5,0) -- ++(0,0.7) -- ++(-1.0,0) -- ++(0,0.8)
	-- ++(-1.6,0) -- ++(0,7.0) -- ++(-3.0,1.0) -- ++(0,5.0) -- ++(3.0,0)
	-- ++(0,-5.0) -- ++(3.1,-3.0) -- ++(0.4,0) -- ++(0,-4.0) -- ++(-0.4,0)
	-- cycle;

% draw screw holes
\draw [holder background] (4.4,16.7) decorate[thread]{ -- ++(3.0,0) }
	-- ++(0,1.6) decorate[thread]{ -- ++(-3.0,0) } -- cycle;
\draw [holder background] (9.4,16.7) decorate[thread]{ -- ++(3.0,0) }
	-- ++(0,1.6) decorate[thread]{ -- ++(-3.0,0) } -- cycle;

% holder cap
\draw [holder cap] (0,4.7) -- ++(2.8,0) -- ++(0,0.5) -- ++(-1.7,0)
	decorate[thread]{ -- ++(0,6.8) } -- ++(-1.1,0) -- cycle;
\draw [holder cap] (16.8,4.7) -- ++(-2.8,0) -- ++(0,0.5) -- ++(1.7,0)
	decorate[thread, decoration = {mirror}]{ -- ++(0,6.8) } -- ++(1.1,0)
	-- cycle;

% O-rings
\draw[oring] (2.3,5.7) circle (0.5);
\draw[oring] (14.5,5.7) circle (0.5);

% rotation axis
%\draw[rotation axis] (8.4,-1.0) -- ++(0,22.0);

% bar scale
\draw[bar scale] (14.8,19) -- node[below] {1\,mm} ++(1.0,0);

\end{tikzpicture}

	\caption[%
		Spinning cell holder.%
	]{%
		\captiontitle{%
			Spinning cell holder.%
		}
		A Teflon plug seals the cell (in blue), the knurled nut (light grey)
		secures the cell to the holder chuck (dark grey) by compressing the o-ring
		(black).
		The cell is attached to the driving motor shaft by M2 screws.
	}
	\label{\figlabel{spinning_cell:drawing}}
\end{figure}

As a driving motor, we used a DC motor (Maxon A-max 110119) controlled by a
homemade power source that could produce from 0\,V to 9\,V at the output.
The power supply was equipped with a digital voltmeter for user convenience.
The capabilities of the motor were measured with an attached fully filled cell.
The values of rotation speed $\omega$ as a function of input voltage $U$ were
fitted by linear dependence
\begin{gather*}
	\omega = a_1U + a_0,\\
	a_1 = (1071 \pm 4) \text{V}^{-1}, a_0 = 2 \pm 20.
\end{gather*}
The dependence has not got any significant constant factor $a_0$, and with the
maximal power of 9\,V accepted by the motor, about 9600\,rpm was achieved,
which was satisfactory for the spinning cell to centrifuge the samples to its
walls completely.

Spinning cell
showed to be impractical for temperature-controlled experiments.
A different approach needed to be devised, but it was still necessary to
minimize the photodecomposition of the samples during measurement.
We decided that we utilize a magnetic stirrer and traditional spectroscopic
1\,cm Raman cells.
We had much experience with recirculating liquid temperature controllers in our
laboratory, but we decided to design a less expensive temperature-controlled
cell holder based on a thermoelectric Peltier device.

The new cell holder consisted of a copper block surrounding the cell
(\figref{thermostated_holder:core_drawing})
with a front hole suitable for backscattering measurements.
There were also two hollows at the top of the block for convenient removal of
the cells.
We also designed an adapter for 5\,mm cells.
The block contained thermocouple Pt100 for measurement of temperature sealed by
thermal grease to ensure good heat transfer.

The copper block was placed inside a Teflon block which served as insulation
(\figref{thermostated_holder:insulation_drawing}).
The insulation layer also contained a magnetic stirrer (Variomag).
The stirrer could be used up to 95\,\textcelsius{}, stir volumes up to 5\,mL on
130 -- 1000\,rpm, and was controlled by an external controller.
The sample cell could be covered inside the holder by a Teflon cover
(\figref{thermostated_holder:stand_drawing}).

% inserting more pdf figures created by Inkscape on one page produces warning
% about multiple pdfs with page group
% I found suggestion for solution here for using qpdf for that but did not try
% to apply it:
% https://tex.stackexchange.com/questions/76273/multiple-pdfs-with-page-group-included-in-a-single-page-warning
\begin{figure}
	\centering
	\ig{1}{results_and_discussion/assets/thermostated_holder_drawing/core}
	\caption[%
		Thermostated cell holder copper core.%
	]{%
		\captiontitle{%
			Thermostated cell holder copper core.%
		}
	}
	\label{\figlabel{thermostated_holder:core_drawing}}
\end{figure}

\begin{figure}
	\centering
	\ig{1}{results_and_discussion/assets/thermostated_holder_drawing/insulation}
	\caption[%
		Teflon insulation of thermostated cell holder.%
	]{%
		\captiontitle{%
			Teflon insulation of thermostated cell holder.%
		}
	}
	\label{\figlabel{thermostated_holder:insulation_drawing}}
\end{figure}

The temperature was maintained by two thermoelectric Peltier modules
(TEC1-12710) chained in series for better performance, where the second module
could be replaced by a copper shim if the performance of one is satisfactory.
The operation of the Peltier modules was controlled by a two-channel PID
controller (Eurotherm 3216) connected to the thermocouple placed inside the
copper block as described above.
The signals of the two channels (heating and cooling) of the PID controller
were converted to a bipolar signal by constructed H bridge so the direction
and intensity of heat transfer through the Peltier modules could be controlled.

An aluminum profile heat sink
(\figref{thermostated_holder:heat_sink_drawing})
with two ventilators (Sukon, 2500\,rpm, 17\,W) was attached to the
thermoelectric modules to increase efficiency.
The aluminum profile was insulated from the copper core block by the Teflon
layer.
A universal aluminum right-angle adapter
(\figref{thermostated_holder:stand_drawing})
mounted the whole sample holder through the heat sink to a three-axis manual
stage (Thorlabs).
The adapter supported securing of different sample holders in various
positions. Overall 3D image of the holder can be seen in
\figref{thermostated_holder:3d}.

\begin{figure}
	\centering
	\ig{1}{results_and_discussion/assets/thermostated_holder_drawing/stand}
	\caption[%
		Teflon cover of sample cell and universal aluminum right-angle
		adapter for thermostated sample holder.%
	]{%
		\captiontitle{%
			Teflon cover of sample cell (on the right) and universal aluminum
			right-angle adapter for thermostated sample holder (on the right).%
		}
	}
	\label{\figlabel{thermostated_holder:stand_drawing}}
\end{figure}

\begin{figure}
	\centering
	\ig{1}{results_and_discussion/assets/thermostated_holder_drawing/heat_sink}
	\caption[%
		Aluminum profile heat sink for the thermostated cell holder.%
	]{%
		\captiontitle{%
			Aluminum profile heat sink for the thermostated cell holder.%
		}
	}
	\label{\figlabel{thermostated_holder:heat_sink_drawing}}
\end{figure}

\begin{figure}
	\centering
	\LOWQUALITY
	\includegraphics[width=.49\columnwidth]%
			{results_and_discussion/assets/thermostated_holder_drawing/3D01}
	\LOWQUALITY
	\includegraphics[width=.49\columnwidth]%
		{results_and_discussion/assets/thermostated_holder_drawing/3D02}
	\caption[%
		The 3D image of thermostated sample holder without thermoelectric
		Peltier modules, thermocouple, and stirrer.%
	]{%
		\captiontitle{%
			The 3D image of thermostated sample holder without thermoelectric
			Peltier modules, thermocouple, and stirrer.%
		}
	}
	\label{\figlabel{thermostated_holder:3d}}
\end{figure}

This design of thermostated sample holder enabled fast control of the
temperature in the range between 5 and 95\,\textcelsius{} of samples with
volumes in the range between 250 (with the adapter to 0.5\,mm sample cells) and
3000\,\g{m}L.
The heating from 5 to 95\,\textcelsius{} took $\sim$15\,min and cooling from 95
to 5\,\textcelsius{} was as fast as 25\,min.

The air conditioning also kept low humidity inside the laboratory, but for
temperatures less then 15\,\textcelsius{} the continuous flow of dry air or
nitrogen was directed to the front surface of the sample cell.

The laser beam, which had horizontal polarization, was guided in a loop using
right angle prisms to maintain the right polarization direction.
Pellin-Broca prism (PB) separated the excitation beam from unwanted light
(for example, from fundamental laser lines).
The unwanted light was guided to a beam blocker (BB).

\emph{M3} and hit the bottom of the sample cell, having polarization
perpendicular to the direction of scattered signal acquisition.
The excitation beam was focused onto the sample by the plano-convex lens
\emph{L1} with a focal length of 100\,mm.
It was calculated that shorter focal lengths would be more optimal for the
apparatus but due to steric restrictions the longer focal length was
chosen because the most practical place for the lens was between M2 and M3.
The excitation laser beam could be attenuated by
the neutral density filters \emph{ND} with a selection of 0.5, 1, 2, 3, and 4
optical densities (Thorlabs), which were placed inside the carousel (Thorlabs)
for easy switching.
We also used an old single-blade shutter that we found in our laboratory.
The shutter was controlled from the CCD port for external shutter by the
WinSpec software provided with the detector.
The shutter was open during measurement and closed the rest of the time
to prevent photodecomposition of samples by the UV laser beam.

We needed to visualize the beam for proper laser beam path setup because the
wavelengths in use are outside of the visible light range.
Business cards showed to be the best equipment for beam visualization because
the paper used for them usually has strong fluorescence.
Excitation beam adjustment procedure uses two apertures \emph{A1}, \emph{A2}
and crosshair reticle printed on the paper fixed to the ceiling of the room.
At first, the kinematic holder of mirror M1 is adjusted, so the beam passes
through aperture A1.
If the beam is blocked by some of the other optical elements on its way to the
ceiling reticle, we can use a business card to detect if it passes through the
center of the aperture A1.
Then the mirror M2 is adjusted so that the beam passes through A2, and finally,
the mirror M3 is adjusted to hit the center of the reticle.
The procedure can be iteratively repeated if the result is not satisfactory but
the good alignment was usually achieved after the first iteration.

Raman scattered light is no longer monochromatic, so we decided to use
reflective optics in the light gathering part of the apparatus to avoid
chromatic aberration.
We selected Cassegrain objective \emph{O} (Newport) with $f_o = 13$\,mm
effective front focal length, back focal length in infinity, 0.4 numerical
aperture, and 9.72\,mm diameter of the output beam.
The elastically scattered light is removed by edge filters \emph{EF} (Semrock).
The rest of the collimated signal is then guided by UV enhanced broadband
aluminum mirror \emph{MS1} (Newport) to focusing off-axis parabolic mirror
\emph{MS2} (Newport) with $f_c = 101.6$\,mm effective focal length to the
spectrograph slit.
The spectrograph has f/6.4 aperture, whereas the focusing lens with the
beam diameter determined by objective has aperture $101.6 / 9.72 = f/10.5$,
which means that we slightly underfill the spectrometer's angle of acceptance.

Highly absorbing samples can be hard to measure in right-angle geometry
because of absorption of excitation light and selfabsorption of the scattered
light \parencite{Shriver1974}.
These problems can be solved by using backscattering geometry.
So we decided to improve the apparatus with an easily switchable
backscattering modality.

We utilized the fact that the Cassegrain objective has a blind spot in
collimation mirror area and placed there a small $0.5 \times 0.5$ right-angle
prism (M4) glued to a homemade holder attached to kinematic stand.
The Cassegrain objective was placed on a long travel manual transition stage so
it could be moved to the position for backscattering, where the laser beam
going up was almost touching the front side of the Cassegrain.
The prism M4 reflected the beam in the optical axis of the Cassegrain into the
sample, see
\figref{backscattering:apparatus_schema}.

\begin{figure}
	\centering
	\begin{tikzpicture}[font=\sffamily]

% settings
\newcommand*{\cellBorderWidth}{3\pgflinewidth}
\newcommand*{\cellLineWidth}{1.5\pgflinewidth}
\definecolor{glassBorderColor}{RGB}{0,128,255}
\definecolor{glassFillColor}{RGB}{230,242,255}
\definecolor{waterFillColor}{RGB}{0,128,255}
\definecolor{hclColor}{RGB}{255,128,0}
\definecolor{unwantedLightColor}{RGB}{153,17,0}
\tikzset{
	clip
}
\tikzset{
	mirror element/.style = {color = black, line width = 2 * \pgflinewidth},
	real laser beam/.style = {color = cyan, line width = 2 * \pgflinewidth},
	laser beam/.style = {real laser beam},
	scattered ray/.style = {color = red!60, line width = 1.5 * \pgflinewidth},
	scattered fill/.style = {fill = red, draw = none, fill opacity = 0.2},
	glass/.style = {color = glassBorderColor, opacity = 0.5,%
		fill = glassFillColor, fill opacity = 0.5},
	sample cell/.style = {color = glassBorderColor, opacity = 0.5,%
		double = glassFillColor, double distance = \cellBorderWidth,
		line width = \cellLineWidth, line cap = rect},
	water fill/.style = {fill = waterFillColor, fill opacity = 0.1},
	mirror surface/.style = {color = black!20, fill = black!10},
	nd filter/.style = {color = black, opacity = 0.2, fill = black,%
		fill opacity = 0.1},
	nd carousel/.style = {color = black, opacity = 0.4, fill = black,%
		fill opacity = 0.2},
	notch/.style = {color = black, opacity = 0.4, fill = black,%
		fill opacity = 0.1},
	aperture/.style = {color = black, line width = 2 * \pgflinewidth},
	aperture filldraw/.style = {color = black!40, fill = black!20,%
		clip even odd rule},
	clip even odd rule/.code = {\pgfseteorule},
	shutter/.style = {nd carousel},
	shutter blade/.style = {dashed, line width = 1.5 * \pgflinewidth,%
		opacity = 0.4},
	hcl/.style = {nd carousel},
	hcl lamp/.style = {color = black!40, line width = 3 * \pgflinewidth,%
		line cap = round},
	hcl ray/.style = {color = black, opacity = 0.2,%
		line width = 1.5 * \pgflinewidth},
	hcl beam/.style = {draw = none, fill = black, opacity = 0.1},
	beam blocker/.style = {mirror element},
	unwanted ray/.style = {color = unwantedLightColor,%
		line width = 1.5 * \pgflinewidth}
};
\clip (-.1,-2.1) rectangle (14,6.6);
\coordinate (shutter) at (11,6);
\newcommand*{\samplePosWidth}{10.9}
\newcommand*{\samplePosHeight}{3}
\newcommand*{\laserUpWidth}{10}
\coordinate (M1) at (13,6);
\coordinate (M2) at (13,\samplePosHeight);
\coordinate (BeamBlocker) at ($ (M2) + (-0.35,1.2) $);
\newcommand*{\beamBlockerWidth}{0.5}
\coordinate (samplePos) at (\samplePosWidth,\samplePosHeight);
\coordinate (laserUp) at (\laserUpWidth,\samplePosHeight);
\newcommand*{\cellWidth}{1};
\coordinate (cassegrainM1Center) at (\samplePosWidth - 0.5,\samplePosHeight);
\newcommand*{\cassegrainMARadius}{1.5}
\coordinate (cassegrainM2Center) at (\samplePosWidth - 0.7,\samplePosHeight);
\newcommand*{\cassegrainMBRadius}{0.4}
\newcommand*{\sqrttwo}{1.414213}
\coordinate (HCL) at (6,0);
\coordinate (AC1) at (6,1.2);  % calibration aperture
\coordinate (LC1) at (6,0.6);  % calibration lens
\coordinate (MS1Edge1) at (4.5,\samplePosHeight + 0.5);
\coordinate (MS1Edge2) at (3.5,\samplePosHeight - 0.5);
\coordinate (MC1Edge1) at (5.5,2.5);
\coordinate (MC1Edge2) at (6.5,1.5);
\coordinate (MC1Edge3) at (5.5,1.5);
\coordinate (MC2FlippedEdge1) at (4.4,2.5);
\newcommand*{\MCBFlippedH}{1.0};  % MC2 height
\newcommand*{\MCBFlippedD}{1.0};  % MC2 depth
\coordinate (parabolaFocus) at (3,0.4);
\coordinate (MS2Edge1) at (4.5,1);
\coordinate (MS2EdgeControl1) at (245:0.5);
\coordinate (MS2Edge2) at (3.5,0);
\coordinate (MS2EdgeControl2) at (17:0.3);
\newcommand*{\apertureOuterRadius}{0.3}
\newcommand*{\apertureInnerRadius}{0.1}
\newcommand*{\pbrot}{305}  % Pellin broca prism rotation

% Pellin-Broca prism path def
% B -----
% |      ----C
% |           \
% |            \
% A ----------- D
\coordinate (PBB) at ($ (M1) + (0.06,0.2) $);
\coordinate (PBA) at ($ (PBB) + (270 + \pbrot:0.6) $);
\coordinate (PBD) at ($ (PBA) + (0 + \pbrot:0.9) $);
\path[name path=PBBtoC] (PBB) -- ++(345 + \pbrot:10);
\path[name path=PBDtoC] (PBD) -- ++(120 + \pbrot:10);
\path[name intersections={of=PBBtoC and PBDtoC, by=PBC}];
\path[name path=PBAB] (PBA) -- (PBB);
\path[name path=PBBC] (PBB) -- (PBC);
\path[name path=PBDA] (PBD) -- (PBA);

% laser
\draw (0,5.5) rectangle ++(3,1) node[pos=.5] {laser};

% laser beam Pellin-Broca coordinates
\path[name path=LToPBAB] (shutter) -- (M1);
\path[name intersections = {of = LToPBAB and PBAB, by = PBAB1}];
\draw[real laser beam] (3,6) -- (shutter);
\path[name path=LToPBBC] (PBAB1) -- ++(340:2);
\path[name intersections = {of = LToPBBC and PBBC, by = PBBC1}];
\path[name path=LToPBDA] (M2) -- (M1);
\path[name intersections = {of = LToPBDA and PBDA, by = PBDA1}];

% Unvanted laser beam
\path[name path=ULToPBBC] (PBAB1) -- ++(347:2);
\path[name intersections = {of = ULToPBBC and PBBC, by = UPBBC1}];
\path[name path=ULToPBDA] (M2) -- ($ (M1) + (-0.12,0) $);
\path[name intersections = {of = ULToPBDA and PBDA, by = UPBDA1}];

% Draw unwanted beam
\draw[unwanted ray] (PBAB1) -- (UPBBC1) -- (UPBDA1)
	-- ($ (BeamBlocker) + (0.08,0) $);

% Draw laser beam
\draw[laser beam] (shutter) -- (PBAB1) -- (PBBC1) -- (PBDA1) -- (M2)
	-- (laserUp);

% neutral density filters
\newcommand*{\ndfilterA}{(3.5,5.8) rectangle ++(0.2,0.4)}
\newcommand*{\ndfilterB}{(3.5,5) rectangle ++(0.2,0.4)}
\draw[nd filter] \ndfilterA;
\draw[nd filter] \ndfilterB;
\draw[nd carousel] (3.45,4.9) rectangle ++(0.3,0.1);
\draw[nd carousel] (3.45,5.4) rectangle ++(0.3,0.4);
\draw[nd carousel] (3.45,6.2) rectangle ++(0.3,0.1);
\node[below] at (3.6,4.9) {ND};

\draw[shutter blade] ($ (shutter) + (0,0.3) $) -- ++(0,-0.6);
\draw[shutter] ($ (shutter) + (-0.1,-0.3) $) -- ++(0.2,0) -- ++(0,-0.4)
	-- ++(-0.2,0) -- cycle;
\node[below] at ($ (shutter) + (0,-0.7) $) {shutter};

% Pellin-Broca prism
\draw[glass] (PBA) -- (PBB) -- (PBC) -- (PBD) -- cycle;
\node[above, shift = {(0.25cm,-0.05cm)}] at (PBB) {PB};
% Mirror 2
\draw[glass] ($ (M2) + (0.2,0.2) $)
	-- node[below, shift = {(0.35cm,0.05cm)}, color = black, opacity = 1.0]{M2}
	($ (M2) + (-0.2,-0.2) $) -- ++(0,0.4) -- cycle;
% Mirror 3 - it should be under the beam so we have to draw it first
%\draw[glass] ($ (samplePos) + (-0.2,0.2) $) rectangle ++(0.4,-0.4);
%\node[above] at ($ (samplePos) + (0,0.5) $) {M3};


% Unwanted frequencies beam blocker
\draw[beam blocker] ($ (BeamBlocker) + (0.5 * \beamBlockerWidth,0) $)
	-- ++(-\beamBlockerWidth,0) node[left] {BB};

% Aperture 1
\draw[aperture] ($ (M2) + (-0.5,\apertureOuterRadius) $)
	node[above]{A1}
	-- ++(0,-\apertureOuterRadius + \apertureInnerRadius);
\draw[aperture] ($ (M2) + (-0.5,-\apertureOuterRadius) $)
	-- ++(0,\apertureOuterRadius - \apertureInnerRadius);
% aperture 2
\draw[aperture filldraw]
	(laserUp) circle (\apertureOuterRadius)
	(laserUp) circle (\apertureInnerRadius);

% Draw laser beam on top
\draw[laser beam] (laserUp) -- (samplePos);

% mirror M4
\draw[glass] ($ (laserUp) + (0.15,0.15) $) rectangle ++(-0.3,-0.3);

% Draw laser focusing lens
\newcommand*{\LARadius}{0.7}
\coordinate (L1Center) at (\laserUpWidth+1.6-\LARadius,\samplePosHeight);
\path[name path=L1Arc, shift={(L1Center)}]
	(270:\LARadius) arc (-90:90:\LARadius);
\path[name path=toL1Arc1] ($ (laserUp)  + (0,0.4) $) -- ++(10,0);
\path[name path=toL1Arc2] ($ (laserUp)  + (0,-0.4) $) -- ++(10,0);
\path[name intersections={of=L1Arc and toL1Arc1, by=L11}];
\mypgfextractangle{\LAAAngle}{L1Center}{L11}
\path[name intersections={of=L1Arc and toL1Arc2, by=L12}];
\mypgfextractangle{\LABAngle}{L1Center}{L12}
\draw[glass] (L11) arc (\LAAAngle:\LABAngle-360:\LARadius) -- ++(-0.05,0)
	-- ($ (L11) + (-0.05,0) $) -- cycle;
\node[above, shift={(0,0.1)}] at (L11) {L1};


%%%%%%%%%%%%%%%%%%%%%
% draw the cassegrain

% calculate intersections with mirror 1 (the objective mirror)
% mirror1 arc
\path[name path=M1arc, shift={(cassegrainM1Center)}]
	(90:\cassegrainMARadius) arc (90:270:\cassegrainMARadius);
% upper top ray
\path[name path=toCassegrainM11] (samplePos) -- ++(135:5);
\path[name intersections={of=M1arc and toCassegrainM11, by=cassegrainM11}];
\mypgfextractangle{\cassegrainMAAAngle}{cassegrainM1Center}{cassegrainM11}
% upper bottom ray
\path[name path=toCassegrainM12] (samplePos) -- ++(165:5);
\path[name intersections={of=M1arc and toCassegrainM12, by=cassegrainM12}];
\mypgfextractangle{\cassegrainMABAngle}{cassegrainM1Center}{cassegrainM12}
% lower top ray
\path[name path=toCassegrainM13] (samplePos) -- ++(195:5);
\path[name intersections={of=M1arc and toCassegrainM13, by=cassegrainM13}];
\mypgfextractangle{\cassegrainMACAngle}{cassegrainM1Center}{cassegrainM13}
% lower bottom ray
\path[name path=toCassegrainM14] (samplePos) -- ++(225:5);
\path[name intersections={of=M1arc and toCassegrainM14, by=cassegrainM14}];
\mypgfextractangle{\cassegrainMADAngle}{cassegrainM1Center}{cassegrainM14}

\draw[scattered ray] (samplePos) -- (cassegrainM11);
\draw[scattered ray] (samplePos) -- (cassegrainM12);
\draw[scattered fill] (samplePos) -- (cassegrainM11)
	arc (\cassegrainMAAAngle:\cassegrainMABAngle:\cassegrainMARadius) -- cycle;
\draw[scattered ray] (samplePos) -- (cassegrainM13);
\draw[scattered ray] (samplePos) -- (cassegrainM14);
\draw[scattered fill] (samplePos) -- (cassegrainM13)
	arc (\cassegrainMACAngle:\cassegrainMADAngle:\cassegrainMARadius) -- cycle;

% draw the cell
\draw[sample cell, water fill]
	($ (samplePos)%
		+ (-0.5 * \cellBorderWidth - \cellLineWidth,-0.5 * \cellWidth) $)
		rectangle ++(\cellWidth,\cellWidth);
\node[below] at ($ (samplePos) + (0.5 * \cellWidth,-0.5 * \cellWidth)%
	+ (-0.5 * \cellBorderWidth,0) + (-0.5 * \cellLineWidth,0) $) {S};

% calculate intersections with mirror 2 (the objective mirror)
% mirror2 arc
\path[
	name path=M2arc, shift={(cassegrainM2Center)}] (90:\cassegrainMBRadius)
		arc (90:270:\cassegrainMBRadius);
% calculate cassegrain mirror2 edges
\path[
	name intersections={of=M2arc and toCassegrainM12, by=cassegrainM2Edge1}];
\mypgfextractangle{\cassegrainMBAAngle}{cassegrainM2Center}{cassegrainM2Edge1}
\path[
	name intersections={of=M2arc and toCassegrainM13, by=cassegrainM2Edge2}];
\mypgfextractangle{\cassegrainMBDAngle}{cassegrainM2Center}{cassegrainM2Edge2}
% upper bottom ray
\path[name path=toCassegrainM22] ($ (samplePos) + (0,0.1) $) -- ++(-10,0);
\path[name intersections={of=M2arc and toCassegrainM22, by=cassegrainM22}];
\mypgfextractangle{\cassegrainMBBAngle}{cassegrainM2Center}{cassegrainM22}
% lower top ray
\path[name path=toCassegrainM23] ($ (samplePos) + (0,-0.1) $) -- ++(-10,0);
\path[name intersections={of=M2arc and toCassegrainM23, by=cassegrainM23}];
\mypgfextractangle{\cassegrainMBCAngle}{cassegrainM2Center}{cassegrainM23}

\draw[scattered ray] (cassegrainM11) -- (cassegrainM2Edge1);
\draw[scattered ray] (cassegrainM12) -- (cassegrainM22);
\draw[scattered fill] (cassegrainM2Edge1)
	arc (\cassegrainMBAAngle:\cassegrainMBBAngle:\cassegrainMBRadius)
		-- (cassegrainM12)
	arc (\cassegrainMABAngle:\cassegrainMAAAngle:\cassegrainMARadius) -- cycle;
\draw[scattered ray] (cassegrainM13) -- (cassegrainM23);
\draw[scattered ray] (cassegrainM14) -- (cassegrainM2Edge2);
\draw[scattered fill] (cassegrainM23)
	arc (\cassegrainMBCAngle:\cassegrainMBDAngle:\cassegrainMBRadius)
		-- (cassegrainM14)
	arc (\cassegrainMADAngle:\cassegrainMACAngle:\cassegrainMARadius) -- cycle;

% to mirror MS1
% path representing the mirror
\path[name path=MS1Path] (MS1Edge1) -- (MS1Edge2);
% intersections with the mirror
% ray 1
\path[name path=toCassegrainM21] (cassegrainM2Edge1) -- ++(-10,0);
\path[name intersections={of=MS1Path and toCassegrainM21, by=MS11}];
% ray 2
\path[name intersections={of=MS1Path and toCassegrainM22, by=MS12}];
% ray 3
\path[name intersections={of=MS1Path and toCassegrainM23, by=MS13}];
% ray 4
\path[name path=toCassegrainM24] (cassegrainM2Edge2) -- ++(-10,0);
\path[name intersections={of=MS1Path and toCassegrainM24, by=MS14}];

\draw[scattered ray] (cassegrainM2Edge1) -- (MS11);
\draw[scattered ray] (cassegrainM22) -- (MS12);
\draw[scattered fill] (cassegrainM2Edge1)
	arc (\cassegrainMBAAngle:\cassegrainMBBAngle:\cassegrainMBRadius) -- (MS12)
		-- (MS11) -- cycle;
\draw[scattered ray] (cassegrainM23) -- (MS13);
\draw[scattered ray] (cassegrainM2Edge2) -- (MS14);
\draw[scattered fill] (cassegrainM23)
	arc (\cassegrainMBCAngle:\cassegrainMBDAngle:\cassegrainMBRadius) -- (MS14)
		-- (MS13) -- cycle;

% draw first mirror of cassegrain
\draw[mirror element]
	(cassegrainM11)
		arc (\cassegrainMAAAngle:\cassegrainMABAngle:\cassegrainMARadius)
			node[left,pos=0.5] {O};
\draw[mirror element]
	(cassegrainM13)
		arc (\cassegrainMACAngle:\cassegrainMADAngle:\cassegrainMARadius);
% mirror 2
\draw[mirror element]
	(cassegrainM2Edge1)
		arc (\cassegrainMBAAngle:\cassegrainMBDAngle:\cassegrainMBRadius);

% draw notch
\draw[notch] ($ (laserUp) + (-3,-0.5) $) rectangle ++(0.2,1);
\node[above] at ($ (laserUp) + (-2.9,0.5) $) {EF};

% Parabolic mirror MS2
\newcommand*{\parabolicMirrorDef}{%
		(MS2Edge2)
		.. controls ($ (MS2Edge2) + (MS2EdgeControl2) $)
			and ($ (MS2Edge1) + (MS2EdgeControl1) $)
		.. (MS2Edge1)
}
\path[name path=MS2Path] \parabolicMirrorDef;
% ray 1
\path[name path=toMS21] (MS11) -- ++(0,-10);
\path[name intersections={of=MS2Path and toMS21, by=MS21}];
% ray 2
\path[name path=toMS22] (MS12) -- ++(0,-10);
\path[name intersections={of=MS2Path and toMS22, by=MS22}];
% ray 3
\path[name path=toMS23] (MS13) -- ++(0,-10);
\path[name intersections={of=MS2Path and toMS23, by=MS23}];
% ray 4
\path[name path=toMS24] (MS14) -- ++(0,-10);
\path[name intersections={of=MS2Path and toMS24, by=MS24}];

\draw[scattered ray] (MS11) -- (MS21);
\draw[scattered ray] (MS12) -- (MS22);
\begin{scope}
	\clip (MS11) -- ($ (MS21) + (0,-1) $) -- ($ (MS22) + (0,-1) $) -- (MS12)
		-- cycle;
	\draw[scattered fill] (MS12) -- \parabolicMirrorDef -- (MS11) -- cycle;
\end{scope}
\draw[scattered ray] (MS13) -- (MS23);
\draw[scattered ray] (MS14) -- (MS24);
\begin{scope}
	\clip (MS13) -- ($ (MS23) + (0,-1) $) -- ($ (MS24) + (0,-1) $) -- (MS14)
		-- cycle;
	\draw[scattered fill] (MS14) -- \parabolicMirrorDef -- (MS13) -- cycle;
\end{scope}

% draw MS1 over all incident rays on that mirror
\draw[mirror element]
	(MS1Edge1) -- node[above,shift={(-0.5cm,-0.1cm)}]{MS1} (MS1Edge2);

% Draw calibration source
\newcommand*{\hclBeamHalfWidth}{0.25}
\draw [hcl] ($ (HCL) + (-0.3,0) $) rectangle ++(0.6,-1);
\node[right] at ($ (HCL) + (0.3,-0.5) $) {HCL};
\draw [hcl lamp] ($ (HCL) + (-0.1 + 0.02, 0) $) --
	++(0.2 - 0.04,0);

% path representing calibration lens surface
\newcommand*{\LCARadius}{0.8}
\coordinate (LC1Center) at ($ (LC1) + (0,\LCARadius) $);
\path[name path=LC1Arc, shift={(LC1Center)}]
	(180:\LCARadius) arc (-180:0:\LCARadius);
% ray 1
\path[name path = toMC11] ($ (HCL) + (-\hclBeamHalfWidth, 0) $) -- ++(0,5);
\path[name intersections = {of = LC1Arc and toMC11, by = LC11}];
\draw[hcl ray] (HCL) -- (LC11);
% ray 2
\path[name path = toMC12] ($ (HCL) + (\hclBeamHalfWidth, 0) $) -- ++(0,5);
\path[name intersections = {of = LC1Arc and toMC12, by = LC12}];
\draw[hcl ray] (HCL) -- (LC12);

% path representing the mirror MC1
\path[name path = MC1Path] (MC1Edge1) -- (MC1Edge2);
% ray 1
\path[name intersections = {of = MC1Path and toMC11, by = MC11}];
\draw[hcl ray] (LC11) -- (MC11);
% ray 2
\path[name intersections = {of = MC1Path and toMC12, by = MC12}];
\draw[hcl ray] (LC12) -- (MC12);

% fill the beam from source to prism MC1
\draw[hcl beam] (HCL) -- (LC11) -- (MC11) -- (MC12) -- (LC12) -- cycle;

% Draw calibration lens
\path[name path=toLC1Arc1] ($ (LC1) + (-0.5,0) $) -- ++(0,1);
\path[name path=toLC1Arc2] ($ (LC1)  + (0.5,0) $) -- ++(0,1);
\path[name intersections={of=LC1Arc and toLC1Arc1, by=LC11}];
\mypgfextractangle{\LCAAAngle}{LC1Center}{LC11}
\path[name intersections={of=LC1Arc and toLC1Arc2, by=LC12}];
\mypgfextractangle{\LCABAngle}{LC1Center}{LC12}
\draw[glass] (LC11) arc (\LCAAAngle:\LCABAngle:\LCARadius) -- ++(0,0.05)
	-- ($ (LC11) + (0,0.05) $) -- cycle;
\node[right] at ($ (LC1) + (0.5,0.1) $) {LC1};

% Calibration aperture 1
\draw[aperture] ($ (AC1) + (-0.5,0) $) -- ($ (AC1) + (-0.3,0) $);
\draw[aperture] ($ (AC1) + (0.3,0) $) -- ($ (AC1) + (0.5,0) $)
	node[right]{AC1};

% path representing the mirror MC2
\path[name path = MC2Path] (MC2FlippedEdge1) -- ++(0,-\MCBFlippedH);
% ray 1
\path[name path = toMC21] (MC11) -- ++(-4,0);
\path[name intersections = {of = MC2Path and toMC21, by = MC21}];
\draw[hcl ray] (MC11) -- (MC21);
% ray 2
\path[name path = toMC22] (MC12) -- ++(-4,0);
\path[name intersections = {of = MC2Path and toMC22, by = MC22}];
\draw[hcl ray] (MC12) -- (MC22);
% fill the beam to prism MC2
\draw[hcl beam] (MC11) -- (MC21) -- (MC22) -- (MC12) -- cycle;

% draw the calibration prism MC1
\draw[glass] (MC1Edge1) -- (MC1Edge2) -- (MC1Edge3) -- cycle;
\node[right] at ($ (MC1Edge1) + (0.5,-0.4) $) {MC1};

% draw the calibration prism MC2
\draw[glass] (MC2FlippedEdge1) -- ++(0,-\MCBFlippedH)
	-- node[below, color = black, opacity=1.0] {MC2}
	++(\MCBFlippedD,0) -- ++(0,\MCBFlippedH) -- cycle;

% draw scattered beam from MS1 to parabolic mirror MS2
\draw[scattered ray] (MS21) -- (parabolaFocus);
\draw[scattered ray] (MS22) -- (parabolaFocus);
\begin{scope}
	\clip (parabolaFocus) -- (MS21) -- ++(1,0) -- ($ (MS22) + (1,0) $) -- (MS22)
		-- cycle;
	\draw[scattered fill] (parabolaFocus) -- \parabolicMirrorDef -- cycle;
\end{scope}
\draw[scattered ray] (MS23) -- (parabolaFocus);
\draw[scattered ray] (MS24) -- (parabolaFocus);
\begin{scope}
	\clip (parabolaFocus) -- (MS23) -- ++(1,0) -- ($ (MS24) + (1,0) $) -- (MS24)
		-- cycle;
	\draw[scattered fill] (parabolaFocus) -- \parabolicMirrorDef -- cycle;
\end{scope}

% draw the parabolic mirror MS2
\draw[mirror element]
	(MS2Edge2)
	.. controls ($ (MS2Edge2) + (MS2EdgeControl2) $)
		and ($ (MS2Edge1) + (MS2EdgeControl1) $)
	.. node[below,shift={(0.5cm,0.1cm)}]{MS2} (MS2Edge1);

\draw (0,0) rectangle ++(3,2.5) node[pos=.5] {spectrograph};

% side view
\begin{scope}[shift={(\samplePosWidth - 2,-1.7)}]

\newcommand*{\samplePosSideWidth}{2}
\newcommand*{\samplePosSideHeight}{2.5}
\newcommand*{\laserUpSideWidth}{1.1}
\coordinate (samplePosSide) at (\samplePosSideWidth,\samplePosSideHeight);
\coordinate (laserUpSide) at (\laserUpSideWidth,0);
\coordinate (cassegrainM1SideCenter)
	at (\samplePosSideWidth - 0.5,\samplePosSideHeight);
\coordinate (cassegrainM2SideCenter)
	at (\samplePosSideWidth - 0.7,\samplePosSideHeight);

% clip the view
\clip (-.05,-.35) rectangle ++(3.3,3.4);
\draw (-.05,-.35) rectangle ++(3.3,3.4);

% laser beam
\draw[laser beam] (4,0) -- (laserUpSide) coordinate (M3Side)
	-- (\laserUpSideWidth,\samplePosSideHeight) coordinate (M4Side)
	-- (samplePosSide);

% mirror 3
%\draw[mirror element] ($ (M3Side) + (-0.3,0.3) $)
%	-- node[above,shift={(0.35cm,-0.05cm)}]{M3} ($ (M3Side) + (0.3,-0.3) $);
\draw[glass] ($ (M3Side) + (-0.2,0.2) $) -- ++(0.4,0)
	node[above right,shift = {(-0.1cm,-0.1cm)}, color = black, opacity = 1.0]{M3}
	-- ++(0,-0.4)
	-- cycle;

% mirror 4
\draw[glass] ($ (M4Side) + (-0.15,-0.15) $) -- ++(0.3,0)
	-- ++(0,0.3) coordinate (M4Side2)
	-- cycle;

% aperture 2
\draw[aperture] ($ (M3Side) + (\apertureOuterRadius,0.8) $)
	node[above, shift={(0.05,0)}]{A2}
	-- ++(-\apertureOuterRadius + \apertureInnerRadius,0);
\draw[aperture] ($ (M3Side) + (-\apertureOuterRadius,0.8) $)
	-- ++(\apertureOuterRadius - \apertureInnerRadius,0);

% draw the cassegrain
% calculate intersections with mirror 1 (the objective mirror)
% mirror1 arc
\path[name path=M1SideArc, shift={(cassegrainM1SideCenter)}]
	(90:\cassegrainMARadius) arc (90:270:\cassegrainMARadius);
% upper top ray
\path[name path=toCassegrainM11Side] (samplePosSide) -- ++(135:5);
\path[name intersections={of=M1SideArc and toCassegrainM11Side,
	by=cassegrainM11Side}];
% upper bottom ray
\path[name path=toCassegrainM12Side] (samplePosSide) -- ++(165:5);
\path[name intersections={of=M1SideArc and toCassegrainM12Side,
	by=cassegrainM12Side}];
% lower top ray
\path[name path=toCassegrainM13Side] (samplePosSide) -- ++(195:5);
\path[name intersections={of=M1SideArc and toCassegrainM13Side,
	by=cassegrainM13Side}];
% lower bottom ray
\path[name path=toCassegrainM14Side] (samplePosSide) -- ++(225:5);
\path[name intersections={of=M1SideArc and toCassegrainM14Side,
	by=cassegrainM14Side}];

\draw[scattered ray] (samplePosSide) -- (cassegrainM11Side);
\draw[scattered ray] (samplePosSide) -- (cassegrainM12Side);
\draw[scattered fill] (samplePosSide) -- (cassegrainM11Side)
	arc (\cassegrainMAAAngle:\cassegrainMABAngle:\cassegrainMARadius) -- cycle;
\draw[scattered ray] (samplePosSide) -- (cassegrainM13Side);
\draw[scattered ray] (samplePosSide) -- (cassegrainM14Side);
\draw[scattered fill] (samplePosSide) -- (cassegrainM13Side)
	arc (\cassegrainMACAngle:\cassegrainMADAngle:\cassegrainMARadius) -- cycle;

% name for M4
\node[above right,shift = {(-0.1cm,-0.1cm)}] at (M4Side2) {M4};

% draw the cell
\draw[sample cell, water fill]
	($ (samplePosSide) + (-0.5 * \cellBorderWidth - \cellLineWidth,%
		-0.5 * \cellBorderWidth - \cellLineWidth) $)
		rectangle ++(\cellWidth,10);

% calculate intersections with mirror 2 (the objective mirror)
% mirror2 arc
\path[
	name path=M2SideArc, shift={(cassegrainM2SideCenter)}] (90:\cassegrainMBRadius)
		arc (90:270:\cassegrainMBRadius);
% calculate cassegrain mirror2 edges
\path[
	name intersections={%
		of=M2SideArc and toCassegrainM12Side, by=cassegrainM2SideEdge1}];
\path[
	name intersections={%
		of=M2SideArc and toCassegrainM13Side, by=cassegrainM2SideEdge2}];
% upper bottom ray
\path[name path=toCassegrainM22Side]%
	($ (samplePosSide) + (0,0.1) $) -- ++(-10,0);
\path[name intersections={%
	of=M2SideArc and toCassegrainM22Side, by=cassegrainM22Side}];
% lower top ray
\path[name path=toCassegrainM23Side]
	($ (samplePosSide) + (0,-0.1) $) -- ++(-10,0);
\path[name intersections={%
	of=M2SideArc and toCassegrainM23Side, by=cassegrainM23Side}];

\draw[scattered ray] (cassegrainM11Side) -- (cassegrainM2SideEdge1);
\draw[scattered ray] (cassegrainM12Side) -- (cassegrainM22Side);
\draw[scattered fill] (cassegrainM2SideEdge1)
	arc (\cassegrainMBAAngle:\cassegrainMBBAngle:\cassegrainMBRadius)
		-- (cassegrainM12Side)
	arc (\cassegrainMABAngle:\cassegrainMAAAngle:\cassegrainMARadius) -- cycle;
\draw[scattered ray] (cassegrainM13Side) -- (cassegrainM23Side);
\draw[scattered ray] (cassegrainM14Side) -- (cassegrainM2SideEdge2);
\draw[scattered fill] (cassegrainM23Side)
	arc (\cassegrainMBCAngle:\cassegrainMBDAngle:\cassegrainMBRadius)
		-- (cassegrainM14Side)
	arc (\cassegrainMADAngle:\cassegrainMACAngle:\cassegrainMARadius) -- cycle;

% to mirror MS1
% ray 1
\draw[scattered ray] (cassegrainM2SideEdge1) -- ++(-10,0);
\draw[scattered ray] (cassegrainM22Side) -- ++(-10,0);
\draw[scattered fill] (cassegrainM2SideEdge1)
	arc (\cassegrainMBAAngle:\cassegrainMBBAngle:\cassegrainMBRadius)
		-- ++(-10,0) -- ($ (cassegrainM2SideEdge1) + (-10,0) $) -- cycle;
\draw[scattered ray] (cassegrainM23Side) -- ++(-10,0);
\draw[scattered ray] (cassegrainM2SideEdge2) -- ++(-10,0);
\draw[scattered fill] (cassegrainM23Side)
	arc (\cassegrainMBCAngle:\cassegrainMBDAngle:\cassegrainMBRadius) --
		++(-10,0) -- ($ (cassegrainM2SideEdge1) + (-10,0) $) -- cycle;

% draw first mirror of cassegrain
\draw[mirror element]
	(cassegrainM11Side)
		arc (\cassegrainMAAAngle:\cassegrainMABAngle:\cassegrainMARadius)
			node[left,pos=0.5] {O};
\draw[mirror element]
	(cassegrainM13Side)
		arc (\cassegrainMACAngle:\cassegrainMADAngle:\cassegrainMARadius);
% mirror 2
\draw[mirror element]
	(cassegrainM2SideEdge1)
		arc (\cassegrainMBAAngle:\cassegrainMBDAngle:\cassegrainMBRadius);

\end{scope}

\end{tikzpicture}

	\caption[%
		Top-view schema of the apparatus in backscattering configuration
		and with side-view inset of the sample space.%
	]{%
		\captiontitle{%
			Top-view schema of the apparatus in backscattering configuration
			and with side-view inset of the sample space.%
		}
		A laser beam is emitted with horizontal polarization.
		It is guided through
			a carousel with neutral density (ND) filters,
			shutter,
			Pellin-Broca prism PB, which separates unwanted frequencies from the
				excitation beam and sends them to the beam blocker BB,
			Right angle prisms in total reflection configuration replaced M2 -- M4,
			and laser focusing lens (L1)
			to the sample cell (S)
			through apertures A1 and A2.
		The scattered light is
			gathered by Cassegrain objective (O),
			reflected by the mirror MS1
			and focused on the entrance slit of spectrograph by parabolic mirror MS2.
			Edge filter (EF) suppresses elastically scattered light.
		The calibration beam from Pt \emph{hollow cathode lamp} (HCL)
			is collimated by calibration beam lens LC1
			and guided by the calibration right-angle prisms (MC1 and MC2) in total
				internal reflection mode
			and focused on the entrance slit of spectrograph by the parabolic mirror
				MS2 the same way as the scattered signal from samples.
		The calibration lamp signal can be attenuated by modifying the size of the
			calibration beam aperture AC1.
		The mirror MC2 was placed on a motorized flip mount to enable easy
			insertion for calibration spectra measurement.
		The prism MC2 is flipped to the position for measurement, and the
			calibration lamp is switched off.
	}
	\label{\figlabel{backscattering:apparatus_schema}}
\end{figure}

The holder for prism M4 is displayed in
\figref{backscattering_holder:drawing}.
It was designed so that if it is secured to the kinematic holder, the axis of
the prism is the same as the axis of the kinematic holder.
The width of the thin part of the holder at the end is the same as the width of
the ribs, which are holding the collimating mirror of the Cassegrain so that
they can be aligned and no scattered light is blocked by the holder.

For the right-angle geometry, the M4 was simply removed, and Cassegrain was
moved backward to focus inside the excitation laser light going up.

\begin{figure}
	\centering
	\begin{tikzpicture}[scale=1, font=\sffamily, >=Latex]

\definecolor{glassBorderColor}{RGB}{0,128,255}
\definecolor{glassFillColor}{RGB}{230,242,255}

\tikzset{
	holder/.style = {color = black, fill = black!20},%
	holder line/.style = {color = black},%
	holder hole/.style = {color = black, fill = black!10},%
	holder clear/.style = {color = black, fill = white},%
	glass/.style = {color = glassBorderColor, opacity = 0.5,%
		fill = glassFillColor, fill opacity = 0.5},
	bar scale/.style = {<->, >=Bar[], line width=1.5*\pgflinewidth},%
}

% top view
\draw[holder] (0,3.65)
	-- ++(0.64,0) coordinate (H11)
	-- ++(2.29,0) coordinate (H21)
	-- ++(1.70,0) coordinate (A11)
	-- ++(0.5,0)
	-- ++(0,-1.16)
	-- ++(3.5,0) coordinate (C11)
	-- ++(1.2,0) coordinate (D11)
	-- ++(0.1,0) coordinate (E11)
	-- ++(0,-0.5)
	-- ++(-0.1,0)
	-- ++(0,0.2) coordinate (D12)
	-- ++(-1.2,0) coordinate (C12)
	-- ++(0,-0.2)
	-- ++(-4.0,0)
	-- ++(0,1.02) coordinate (A12)
	-- ++(-1.7,0)
	-- ++(-2.29,0)
	-- ++(-0.64,0)
	-- cycle;
\draw[holder hole] ($ (H11) + (-0.24,0) $) -- ++(0.48,0) -- ++(0,-0.32)
	-- ++(-0.09,0) -- ++(0,-0.32) -- ++(-0.3,0) -- ++(0,0.32) -- ++(-0.09,0)
	-- cycle;
\draw[holder hole] ($ (H21) + (-0.24,0) $) -- ++(0.48,0) -- ++(0,-0.32)
	-- ++(-0.09,0) -- ++(0,-0.32) -- ++(-0.3,0) -- ++(0,0.32) -- ++(-0.09,0)
	-- cycle;
\draw[holder line] (A11) -- (A12);
\draw[holder line] (C11) -- (C12);
\draw[holder line] (D11) -- (D12);

% right angle prism
\draw[glass] (E11) rectangle ++(0.5,-0.5);

% side view
\draw[holder] (0,1.28)
	-- ++(0.64,0) coordinate (SH11)
	-- ++(2.29,0) coordinate (SH21)
	-- ++(1.70,0)
	-- ++(0,-0.39) coordinate (SA11)
	-- ++(0.5,0) coordinate (SB11)
	-- ++(3.5,0)
	-- ++(0,-0.2) coordinate (SC11)
	-- ++(1.2,0) coordinate (SD11)
	-- ++(0,0.2)
	-- ++(0.1,0) coordinate (SE11)
	-- ++(0,-0.5)
	-- ++(-0.1,0)
	-- ++(0,0.2) coordinate (SD12)
	-- ++(-1.2,0) coordinate (SC12)
	-- ++(0,-0.2)
	-- ++(-3.5,0) coordinate (SB12)
	-- ++(-0.5,0) coordinate (SA12)
	-- ++(0,-0.39)
	-- (0,0) -- cycle;
\draw[holder hole] ($ (SH11) + (0,-0.64) $) circle (0.48);
\draw[holder clear] ($ (SH11) + (0,-0.64) $) circle (0.3);
\draw[holder hole] ($ (SH21) + (0,-0.64) $) circle (0.48);
\draw[holder clear] ($ (SH21) + (0,-0.64) $) circle (0.3);
\draw[holder line] (SA11) -- (SA12);
\draw[holder line] (SB11) -- (SB12);
\draw[holder line] (SC11) -- (SC12);
\draw[holder line] (SD11) -- (SD12);

% right angle prism
\draw[glass] (SE11) rectangle ++(0.5,-0.5);

% bar scale
\draw[bar scale] (9.3,3.5) -- node[below] {1\,cm} ++(1.0,0);

\end{tikzpicture}

	\caption[%
		Backscattering holder.%
	]{%
		\captiontitle{%
			Backscattering holder.%
		}
		The top view is on top; a side view is on the bottom.
	}
	\label{\figlabel{backscattering_holder:drawing}}
\end{figure}
