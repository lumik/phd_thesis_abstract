\section{Optimization of the experiment}

Optimal measurement conditions are one of the requirements for high-quality
scientific results.
Many adjustable parameters can influence the measurement quality, and their
optimal values highly depend on the samples under investigation.
The further analysis focuses on optimizing measurement parameters of nucleic
acids using 244 and 257\,nm excitation wavelengths.
PolyU was chosen as the sample for these experiments because uracil is known to
be the most susceptible base for photodecomposition, and our further
experiments were conducted on oligo and polynucleotides, and thus polyU is a
better model molecule than, for example, mononucleotide UMP.
The main parameters which needed to be determined were excitation laser power,
length of accumulation, sample volume, and concentration.
Moreover, the optimal sample rotation speed needed to be found for the
experiments using the spinning cell.

Quality of Raman measurement in dependence on excitation laser power length,
sample concentration, sample volume and rotation speed was quantified by means
of sample lifetimes.
The analysis was performed on the polyU integral intensity represented by
its band at 1231\,\icm{} compared to the intensity of the cacodylate band at
607\,\icm{} to account for the excitation laser power and sample adjustment
variations during the measurement.
The intensity was estimated both from raw spectra and spectra with subtracted
background.

\Figref{power_optim:triplexes_pU}
clearly shows that the polyU band in raw spectrum also contains a shoulder at
1248\,\icm{}, and is overlapping with the cacodylate band at 1276\,\icm.
Therefore, the band was modeled by the shape function
(\eqnref{band_intensities:shape})
with three components, but its intensity was calculated as a sum of the pulyU
bands without the cacodylate one.
The cacodylate band was modeled as having two components, at 607 and 640\,\icm,
but only the first one was used for the integral intensity calculation.

\begin{figure}
	\centering
	\input{results_and_discussion/assets/power_optimization_triplexes/%
		power_optimization_triplexes_pU}
	\caption[%
		Example UV RR spectrum of polyU dissolved in cacodylate buffer.%
	]{%
		\captiontitle{%
			Example UV RR spectrum of polyU dissolved in cacodylate buffer.%
		}
		The fits of shape functions
		(\eqnref{band_intensities:shape})
		to the bands of polyU and cacodylate used in intensity estimation fit are
		marked by green, and red lines, respectively, and the areas used for
		intensity calculation are filled with the corresponding colors.
	}
	\label{\figlabel{power_optim:triplexes_pU}}
\end{figure}

The time dependence of the intensity of polyU band at 1231\,\icm{} (normalized
to the intensity of cacodylate band at 607\,\icm) was modeled by the
exponential decay curve
\begin{equation}
	I = I_0 \text{e}^{-\lambda t} + b,
	\label{\eqnlabel{power_optim:decay}}
\end{equation}
where $I_0$ is initial intensity, $\lambda$ is the decay constant, and $b$ is
the baseline constant.
The normalized results (subtracted by $b$) can be seen in
\figref{power_optim:triplexes}.

\begin{figure}
	\centering
	\input{results_and_discussion/assets/power_optimization_triplexes/%
		power_optimization_triplexes}
	\caption[%
		Decrease of the integral intensity of polyU band at 1231\,\icm{} for
		different excitation laser powers in raw spectra.%
	]{%
		\captiontitle{%
			Decrease of the integral intensity of polyU band at 1231\,\icm{} for
			different excitation laser powers in raw spectra.%
		}
		It was normalized to the integral intensity of the cacodylate band at
		607\,\icm{}, which was used as the internal intensity standard.
		The values were fitted by exponential decay curves
		\eqnref{power_optim:decay}
		and the baseline constant $b$ from the fit was subtracted from the plots.
	}
	\label{\figlabel{power_optim:triplexes}}
\end{figure}

The lifetimes $\tau$ can be calculated from decay constants $\lambda$ from
\eqnref{power_optim:decay}
\begin{equation}
	\tau = \frac{1}{\lambda}.
	\label{\eqnlabel{power_optim:lifetime}}
\end{equation}
We can also calculate total accumulated relative energy from the decay curve
by
\begin{equation*}
	E_{0,\text{r}}
		= \int_0^{\infty}{I_0 \text{e}^{-\lambda t}\text{d}t}
		= \frac{I_0}{\lambda}
\end{equation*}
and energy accumulated from time $T_0$ by
\begin{equation*}
	E_r	= \int_{T_0}^{\infty}{I_0 \text{e}^{-\lambda t}\text{d}t}
		= \frac{I_0}{\lambda} \text{e}^{-\lambda T_0}.
\end{equation*}
Energy fractions can then be calculated by dividing the relative energies by
the maximal total relative energy. The resulting lifetimes, total accumulated
energy fractions $E_0$, and energy fractions $E$ accumulated from time
$T = 60\pm20$\,s, which was regarded as reasonable approximate for the time
needed for adjustment of the sample before measurement could start.

Further investigation of the spectral decay behavior shows that this simple
approach does not reveal that a single decay curve cannot model such a complex
system.
This effect is clearly visible if we subtract the cacodylate buffer background
from the spectra.
The resulting spectrum can be seen in
\figref{power_optim:triplexes2_pU}.

\begin{figure}
	\centering
	\input{results_and_discussion/assets/power_optimization_triplexes2/%
		power_optimization_triplexes2_pU}
	\caption[%
		Example UV RR spectrum of polyU dissolved in cacodylate buffer with
	  the cacodylate buffer background subtracted.%
	]{%
		\captiontitle{%
			Example UV RR spectrum of polyU dissolved in cacodylate buffer with
			the cacodylate buffer background subtracted.%
		}
		The fit of shape function
		(\eqnref{band_intensities:shape})
		to the band of polyU used in intensity estimation fit is marked by a green
		line together with the area used for the intensity calculation.
	}
	\label{\figlabel{power_optim:triplexes2_pU}}
\end{figure}

The polyU band at 1230\,\icm{} can then be modeled directly without the
interference of the cacodylate band at 1276\,\icm{}.
It can be seen from the polyU band intensity dependence plot in
\figref{power_optim:triplexes2}
that the background subtraction process slightly obscures the intensity
normalization so that the fast decay curves cannot be modeled here.
However, we can also see a slower decay trend which is not evident from the fit
to the intensities of the polyU band calculated from the spectra without
subtracted background.
This is caused by the fact that it is harder to reliably decouple
the contribution of the overlapping polyU and cacodylate bands with the
lowering polyU band intensity.

\begin{figure}
	\centering
	\input{results_and_discussion/assets/power_optimization_triplexes2/%
		power_optimization_triplexes2}
	\caption[%
		Decrease of the integral intensity of the polyU band at 1231\,\icm{}
		for different excitation powers in background-corrected spectra.%
	]{%
		\captiontitle{%
			Decrease of the integral intensity of the polyU band at 1231\,\icm{}
			for different excitation powers in background-corrected spectra.%
		}
		The intensity was normalized to the subtracted spectrum of cacodylate
		buffer, which was used as the internal intensity standard.
		The values were fitted by exponential decay curves
		\eqnref{power_optim:decay}.
		The baseline constant $b$ from the fit was subtracted from the plots.
	}
	\label{\figlabel{power_optim:triplexes2}}
\end{figure}

We can see that the system is complex, and our current methods cannot correctly
analyze the underlying processes.
Photoproducts can influence the resulting spectra, and therefore we
decided to use relatively lower excitation powers like 1\,mW or 0.5\,mW even
though the higher excitation power could probably give higher quality spectra.

Next conclusion was that the lifetime seemed to depend almost linearly on
the concentration, proportional to the number of nucleotides in the sample.
This means that in our experimental configuration, all the
excitation laser energy is absorbed in the sample for investigated, and
the number of photodecomposed molecules is proportional to the absorbed
energy in this concentration range.
Only in the samples with extremely low concentrations, all laser power is not
absorbed, and therefore the lifetime is longer.

It also implies that the excitation laser power density does not significantly
influence the photodecomposition process in the used range of the laser powers.

This analysis means that samples can endure longer measurements on higher
concentrations.
On the other hand, higher concentrations are also less cost-effective because
they require larger amounts of samples and it is harder to adjust focus for
them.
The spectra are also more influenced by the signal of the sample cell
because the laser beam needs to be focused closer to the cell wall and floor.
1000\,\g{m}M concentration was the highest practical value; higher
concentrations had excessive absorbance and were hard to set up for the
measurement.
It can also be beneficial for some measurements to measure samples with the
same concentration as in UV absorption measurements so that these two can
complement each other under the same measurement conditions.
Therefore the lower concentrations than 1\,mM were usually chosen in this
study.

The results of sample volume optimization supported the hypothesis that the
sample photodecomposition is inversely proportional to the number of
illuminated molecules with sufficient stirring.
We, therefore, tried to use fully-filled cuvettes by samples (3\,mL in the case
of the thermostated sample cell and 150\,\g{m}L for the spinning cell).

Also, a fast rotation seemed better than the slower rotation with a
stronger signal and no impact on the sample lifetime.
Therefore we decided to use the maximal rotation speed in all the experiments
which used the spinning sample cell.

Lastly, also length of accumulation was optimized.
A Raman spectrum is usually acquired as a consecutive series of spectra,
further called \emph{frames}, taken at the same experimental conditions and
with the constant accumulation time.
We decided to assess the SNR to estimate the quality of the measured spectra.
It was found that longer single-frame accumulation times give better quality
spectra in these experimental conditions.
The difference is larger in the low-frequency SNR even though the
high-frequency SNR difference is also significant.
Regarding the disadvantages of the
longer accumulation discussed in the thesis and the speed of
the sample degradation with lifetimes in the range of minutes, we decided that
the optimal accumulation time is 60\,s.

The spinning cell can be used not only for measurements in the right-angle
geometry, but also for measurements in the backscattering configuration.
We observed two sharp lines at 1555 and 2329\,\icm{} in these measurements
when very small sample volumes were used.
It can be seen, for example, in the spectrum of the samples extracted from alga
amphidinium in
\figref{artefact:artefact_amphidinium}.
50\,\g{m}l of sample was measured in the spinning cell in backscattering
geometry with 10\,mW of 257\,nm excitation laser.

\begin{figure}
	\centering
	\input{results_and_discussion/assets/artefact/%
		artefact_amphidinium}
	\caption[%
		Spectrum of extract from alga amphidinium in the spinning cell in
		backscattering geometry with 257\,nm excitation laser.%
	]{%
		\captiontitle{%
			Spectrum of extract from alga amphidinium in the spinning cell in
			backscattering geometry with 257\,nm excitation laser.%
		}
	}
	\label{\figlabel{artefact:artefact_amphidinium}}
\end{figure}

Further investigation and searching through the databases and atlases of
Raman spectra identified the cause of the spectra as molecules of gas \ch{O2}
at 1556\,\icm{} and \ch{N2} at 2330\,\icm{}.
It means that the signal is also gathered from the central cylinder of the air
inside the spinning cell resulting in the contribution of gas \ch{O2} and
\ch{N2} in the spectra.
