\section{Interpretation of UV resonance Raman spectra of nucleic acids}
\label{interpretation}

% setting global parameters for tables
\newlength{\assignwnl}
\settowidth{\assignwnl}{0000}
\newlength{\assignwnil}
\settowidth{\assignwnil}{(000)}
\newlength{\assignwnspl}
\setlength{\assignwnspl}{0.2cm}
\newlength{\assigntabrowindent}
\setlength{\assigntabrowindent}{.7em}

%%%

One of the primary goals of this thesis was to summarize the current knowledge
about resonance Raman measurements of nucleic acids and critically analyze the
results regarding own measurements of NA model molecules to
create a reliable interpretation table for further NA studies via UV RRS.
Resonance Raman spectra of building blocks of nucleic acids AMP (250\,\g{m}M),
TMP (500\,\g{m}M), GMP (375\,\g{m}M) and polyC (500\,\g{m}M) and melting their
simple complexes poly(dAdT) (500\,\g{m}M), polyG (375\,\g{m}M) and
polyG$\cdot$polyC (500\,\g{m}M) were measured for that purpose.
The concentrations of the samples are in nucleic bases.

The Raman spectra were excited at 244\,nm and the samples were measured at
20\,\textdegree{}C or in the range from 5\,\textdegree{}C to 95\,\textdegree{}C
with the 5\,\textdegree{}C step.

Melting curves of the spectra were analyzed by means of PCA, and the spectrum
at 5\,\textdegree{}C and 95\,\textdegree{}C was constructed from the first 4
principal components.

The results of the measurements normalized to the same concentrations of
nucleobases are displayed in
\cref{%
	\figlabel{interpretation:at},%
	\figlabel{interpretation:gc}%
}.
It means that the spectra of complexes that contain two types of nucleobases
were doubled in intensity to make them directly comparable.
The spectra were decomposed to the Gaussian and Lorentzian curve combinations.
The bands' positions resulting from the fit shown in the figures with their
intensities were estimated.
The precision of the well-resolved bands is better than 2\,\icm{} in position
and 10\,\% in intensity but significantly lowers on overlapped bands.
Finally, the available literature was searched for the band assignments and
added to the interpretation tables (see full text of the work).
This information was then used in the next scientific work in this thesis, and
the table was also enhanced for the information from these works.

\begin{figure}
	\centering
	\input{results_and_discussion/assets/interpretation/at/%
		interpretation_at}
	\caption[%
		Spectra of AMP, TMP and poly(dAdT) obtained at 5\,\textdegree{}C and
		95\,\textdegree{}C.
	]{%
		\captiontitle{%
			Spectra of AMP (a), TMP (b), poly(dAdT) (c) and (d) obtained at
			5\,\textdegree{}C and 95\,\textdegree{}C, respectively.
		}%
		The spectra were excited at 244\,nm, and spectral bands were decomposed
		according to
		\eqnref{band_intensities:shape}
		with the experimentally determined
		$c_\text{L} = 0.39$.
		Each spectrum is marked by a scaling factor on the right side.
		The intensities were multiplied by this factor for better readability.}
	\label{\figlabel{interpretation:at}}
\end{figure}

\begin{figure}
	\centering
	\input{results_and_discussion/assets/interpretation/gc/%
		interpretation_gc}
	\caption[%
			Spectra of GMP, polyC, poly(G) and polyG$\cdot$polyC obtained at
			5\,\textdegree{}C and 95\,\textdegree{}C.
	]{%
		\captiontitle{%
			Spectra of GMP (a), polyC (b), poly(G) (c) and (d) and
			polyG$\cdot$polyC (e) and (f) obtained at 5\,\textdegree{}C and
			95\,\textdegree{}C, respectively.
		}%
		The spectra were excited at 244\,nm, and spectral bands were decomposed
		according to
		\eqnref{band_intensities:shape}
		with the experimentally determined
		$c_\text{L} = 0.39$.
		Each spectrum is marked by a scaling factor on the right side.
		The intensities were multiplied by this factor for better readability.}
	\label{\figlabel{interpretation:gc}}
\end{figure}

To better understand the spectra of complex NA, the spectra of single
mononucleotides AMP, TMP and GMP were measured first.
Also, the spectrum of polyC, which contains a just single nucleotide, was
added.
These spectra were then correlated with the spectra of poly(dAdT), polyG, and
polyG$\cdot$polyC where poly(dAdT), and polyG$\cdot$polyC are known to form the
double-helical structure on lower temperatures
\parencite{Benevides2005}
and polyG forms quadruplex structure with sufficient concentration of kations
in the solution
\parencite{Simard1994}.

The last step was to measure the melting curves of the complexes and get
their spectra above the melting temperature. The safe temperature for all the
used complexes was estimated as 95\,\textdegree{}C.

The changes in the Raman spectra between high- and low-temperature spectrum
can then be used to characterize the Raman bands because the changes
reflect the structural transition between the unfolded and folded forms.

Published calculations of the vibration modes of mostly simple molecules like
methylated bases, nucleosides, and nucleotides and experimental studies on
structural information of NA were reviewed and used to construct the
interpretation tables.
Even though there were some differences between the published data, the
majority of Raman bands in the measured area could be assigned to fundamental
transitions of vibrational modes localized at least in large part on the
nucleobases.
