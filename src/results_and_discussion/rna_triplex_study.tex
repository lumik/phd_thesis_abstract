\section[\texorpdfstring{%
    PolyA and PolyU Complexes, effect of Mg\textsuperscript{2+}
}{%
    PolyA and PolyU Complexes, effect of
		Mg\texttwosuperior\textplussuperior
}]{%
    PolyA and PolyU Complexes, effect of Mg\textsuperscript{2+}
}%

UV RRS was used to study formation of complexes in mixtures of complementary
polynucleotides, polyA and polyU.
Special attention was given to the role of magnesium ions.
It was known that even relatively low concentrations of \ch{Mg^{2+}}
ions can stabilize the polyU$\cdot$polyA$\cdot$polyU triple helical structure,
even when the concentrations of polyU and polyA are equal, i.e., their
stoichiometry is ideal for formation of duplexes.
This effect was observed by UV absorption
\parencite{%
	Kankia2003,%
	Sorokin2003%
},
isothermal titration calorimetry, ultrasound velocimetry and densimetry
\parencite{Kankia2003},
as well as off-resonance Raman spectroscopy
\parencite{Herrera2010}.
Our work widened the NA concentration range of the \ch{Mg^{2+}} titrated
polyA + polyU mixture and provided well-resolved UVRR spectra of individual
components, i.e., single stranded polyA, single stranded polyU,
polyA$\cdot$polyU duplex, and polyU$\cdot$polyA$\cdot$polyU triplex, mutually
aligned in absolute intensity
(\figref{rna_triplex:spectra}).

\begin{figure}[p]
	\centering
	\input{results_and_discussion/assets/rna_triplex/%
		rna_triplex_spectra}
	\caption[%
		Resonance Raman spectra (244\,nm excitation) of polyA, polyU and
		their 1:1 and 1:2 mixtures in aqueous solution.
	]{%
		\captiontitle{%
			Resonance Raman spectra (244\,nm excitation) of a) polyA, b) polyU and
			their c) 1:1 and d) 1:2 mixtures in aqueous solution.
		}%
		The spectra are normalized to the same concentration of phosphate units.
	}
	\label{\figlabel{rna_triplex:spectra}}
\end{figure}

A detailed analysis of the spectral changes was performed with respect to known
modifications in the vicinities of vibrating nucleoside moieties on the one
hand and interpretations of particular vibrational modes proposed in the
literature on the other hand.
Due to the reliability of both the Raman band positions and intensities in our
spectra, some interpretations found in the literature were confirmed and others
excluded.

The effect of \ch{Mg^{2+}} ions to the triplex formation was studied by means of
UV RRS titration experiment.
Magnesium ions were gradually added to an equimolar mixture of polyA and polyU
that originally contained only polyA$\cdot$polyU duplexes.
Each titration series (four series in total) was analyzed by the SVD algorithm,
and the second spectral component always showed a continuous linear increase in
its respective coefficients, following the increase of \ch{Mg^{2+}}
concentration.
The spectral shape of this component was almost identical with a difference
spectrum calculated from the spectra in
\figref{rna_triplex:spectra}
for the expected transition:
\begin{equation*}
	\ch{
		2 (polyA$\cdot$polyU) -> polyU$\cdot$polyA$\cdot$polyU + polyA
	}.
\end{equation*}

The only differences seemed to reflect some interaction of \ch{Mg^{2+}} ions
with the polyA incorporated into the triple helix and/or with the free polyA
strand.
This very good agreement proved the correctness of the spectral normalization
used for the spectra shown in
\figref{rna_triplex:spectra},
as the SVD results reflected the changes in the spectral shape independently of
the intensity scaling.
It was then possible to analyze the \ch{Mg^{2+}} titration by a fit to a sum of
the spectra from the set in
\figref{rna_triplex:spectra},
i.e., the spectra of polyA$\cdot$polyU duplex, polyU$\cdot$polyA$\cdot$polyU
triplex, and single strands.
The resulting duplex and triplex proportions indicated that the reaction
equilibrium followed an approximately linear dependence on the concentration of
\ch{Mg^{2+}} ions.
The equilibrium between the duplex and triplex form was reached for 12\,mM
concentration of magnesium ions.

Our results extend the data already published in a few works devoted to the
phenomenon of \ch{Mg^{2+}} induced triplex formation in an equimolar solution
of polyA and polyU.
The published data were obtained, however, for very different polynucleotide
concentrations.
Considering these results, we concluded that the mechanism of the \ch{Mg^{2+}}
function in the triple helix formation is complicated and comprises both
specific and nonspecific interactions.
The complete study has been published in \textcite{Klener2015}.
