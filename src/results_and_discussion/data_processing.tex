\section{Development of data processing methods}
\label{data_processing}

High-quality spectra are needed for advanced multivariate statistic methods
used for the measurement analysis.
It is essential to have sufficiently sensitive apparatus which can provide a
reliable signal.
However, raw Raman spectra, which come from the measurement, usually contain
many components, and only one of them belongs to the analyte signal.
The following sections describe the process of transforming the raw spectrum
to the spectrum ready for the advanced result analysis, analysis
of spectra of multicomponent system, minimization algorithm used for the
data analysis, and band intensity estimation overview.

Among the major artifacts observed in Raman spectra is signal caused by cosmic
rays, which is characteristic by sharp lines usually impacting only a few
pixels of the CCD detector.
\emph{Spycor} program
\parencite{Spycor2018},
was created as part of this thesis and is based on a comparison of consecutive
spectra where it counts on the fact that during macroscopic Raman measurement,
the spectra change only slowly with time and that the consecutive spectra are
similar.

Raman spectra of liquid samples usually consist of many components, including
the signal of the analyte under investigation and the spectrum of the solvent.
In the case of absorbing analyte, the spectrum
is acquired from the near proximity of the sample cell wall, and so it
usually contains a signal of quartz.
The resulting spectrum can also contain unspecific background from different
sources like fluorescence from residuals from chemical synthesis and elastic
scattering of light from different light sources in the laboratory.

Therefore the background subtraction procedure took several steps.
Each spectrum was acquired as a series of frames, and all frames were stored
for further analysis.
The spectrum of solvent, buffer, and sample cell together with the intensity
normalization to the subtracted buffer signal was performed on each frame in
the first step using the \emph{Bgcor} program
\parencite{Bgcor2017}
(written in \cite{Matlab}),
developed as part of this thesis. The unspecific background was then removed
from each spectrum using background subtraction in frames decomposed to
spectral components by PCA
\parencite{Palacky2011}.
The thirds step consisted of extrapolation to zero time, where PCA decomposed
the frames to spectral components, and the spectrum at zero time
was reconstructed from significant ones.
The last optional step performed only when a series of spectra was
taken, for example, temperature-dependent measurement was to subtract the
background once more using PCA the same way as in step two but currently
applied to the whole series of the extrapolated spectra from the third step.
This section focuses mainly on the first step -- the subtraction of the
solvent, buffer, and sample cell, together with the intensity normalization.

Further, we adopted in this thesis \emph{Principal component analysis} (PCA,
\cite{%
	Wold1987,%
	Malinowski2002%
}),
which reduces the measured spectra to several spectral profiles (loadings) and
scores indicating each profile's portion in the measured spectra.
The scores can then be fitted by a function based on an underlying chemical
model, and for example, thermodynamic parameters for structural transitions
can be estimated in temperature-dependent measurement
\parencite{Nemecek2013}.
But the fit is usually not linear in the system parameters.
Nonlinear minimization is usually much more expensive because it utilizes
iterative methods of searching for the minimum.
They are also susceptible to finding only local minima.
On the other hand, linear least squares regression is “just” solution of a set
of normal equations
\parencite[p.~671]{NumericalRecipes}.
It means that the nonlinear iterative minimization algorithm can be applied
to $b_{k,m}$ values, the $a_{k,l_k}$ values are estimated by linear least
squares for the given $c_k(t_i)$ which are determined by the values of
$b_{k,m}$.
We made use of this fact and divided the fit to two parts where linear
parameters were estimated by linear fit in each step of iterative nonliner
fit which minimized only the nonlinear parameters.
This approach significantly reduced dimension of the nonlinear fit and
improved its numerical stability.

In this thesis, many data treatment approaches needed efficient nonlinear
minimization algorithm (background subtraction, estimation of band intensities,
estimation of parameters of chemical model from spectral series).
The Levenberg-Marquardt method
\parencite{Marquardt1963}
was used as a basis for the minimization algorithm used in this thesis, and
slight modifications were applied to improve its performance.

\label{band_intensities}

One of the important spectroscopic information is the integral intensity of
particular Raman bands from the spectrum.
The band's shape is usually modeled by Gaussian or Lorentzian curves or as
their combination.
The combination of the Lorentzian $\func{L}$ and Gaussian $\func{G}$ curve can
be expressed as
\begin{equation}
	\func{S}(\wn; I_\text{m}, \mu, \sigma) =
		c_\text{L} \cdot \func{L}(\wn; I_\text{m}, \mu, \sigma)
		+ (1 - c_\text{L}) \cdot \func{G}(\wn; I_\text{m}, \mu, \sigma),
	\label{\eqnlabel{band_intensities:single_shape}}
\end{equation}
where $c_\text{L}$ is the Lorentzian curve fraction coefficient, $I_m$ is the
height, $\mu$ is the band position, $\sigma$ is the Gaussian root mean square
width, and $\wn$ is the wavenumber.

Raman bands in a typical Raman spectrum of complex samples are overlapped.
We solved this problem by modeling the band as a combination of more
band-shape functions from
\eqnref{band_intensities:single_shape}
\begin{equation}
	\func{S}(\wn; I_{\text{m},1..n}, \mu_{1..n}, \sigma_{1..n}) =
		\sum_{i = 1}^n = 	\func{S}_i(\wn; I_{\text{m},i}, \mu_i, \sigma_i),
	\label{\eqnlabel{band_intensities:shape}}
\end{equation}
where $n$ is the number of the overlapping bands.
