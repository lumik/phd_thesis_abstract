\chapter*{Preface}
\addcontentsline{toc}{chapter}{Preface}

Despite the decades of intensive research, \emph{nucleic acids} (NA) represent
still a permanent object of the structural studies, which focus on
characterizing NA local geometric arrangement, its formation, and collapse
depending on the internal (nucleobase sequence) and external (environmental
conditions, intermolecular interactions) parameters.
The reason is the rich structural polymorphism of nucleic acids, which are in
addition to the classic Watson-Crick duplex capable of adopting various
non-canonical structures, such as hairpins, cruciforms, triplexes,
quadruplexes, or i-motifs.
It has also been shown that the processes of the expression of genetic
information, of its modification and transfer are controlled through specific
nucleic acid-protein interactions.
However, detailed molecular mechanisms of these fundamental biological
processes are often unknown
\parencite{%
	Bansal2014,%
	Kaushik2016,%
	Yamamoto2021%
}.

A variety of experimental techniques has been used to investigate the
structural arrangement of nucleic acids in their natural environment.
Many of them were gradually developed to achieve credible and
well-interpretable results.
One of the relatively new techniques considered as perspective is the
resonance Raman scattering excited in the ultraviolet spectral region (UV RRS).
UV RRS of NA and their components have been measured since the beginning of
the seventies.
In contrast to the original optimistic expectations, this method is though
currently applied in nucleic acid studies only rarely.

The aim of my doctoral work was to implement the UV RRS method in the Division
of Biomolecular Physics of the Institute of Physics CU, where they had almost
no experience with the method.
The thesis brings a detailed description of the several-year effort,
particularly construction of the spectrometer, solving numerous methodological
problems and application of the built methodology in several structural studies
of nucleic acids.
The first part of the thesis is devoted to the review of published works
concerning UV RRS of nucleic acids, which I used as the starting point for
dealing with the tasks of my doctoral work.
