% !!! don't forget to also edit xmp metadata

%\def\ThesisVersion{}
\def\ThesisVersion{Created: \DTMcurrenttime, \DTMtoday\\}

% Thesis title in English (exactly as in the formal assignment)
\def\ThesisTitle{%
    Study of nucleic-acid structure and interactions by resonance Raman
		scattering}

% Author of the thesis
\def\ThesisAuthor{Mgr. Jakub Klener}

% Year when the thesis is submitted
\def\YearSubmitted{2021}

% Name of the department or institute, where the work was officially assigned
% (according to the Organizational Structure of MFF UK in English,
% or a full name of a department outside MFF)
\def\Department{Institute of Physics}

% Is it a department (katedra), or an institute (ústav)?
\def\DeptType{Department}

% Thesis supervisor: name, surname and titles
\def\Supervisor{Prof. RNDr. Josef Štěpánek, CSc.}

% Supervisor's department (again according to Organizational structure of MFF)
\def\SupervisorsDepartment{Institute of Physics}

% Study programme and specialization
\def\StudyProgramme{Physics}
\def\StudyBranch{%
    \vtop{\hbox{\strut Biophysics, Chemical and Macromolecular}
		\hbox{\strut Physics}}}

% An optional dedication: you can thank whomever you wish (your supervisor,
% consultant, a person who lent the software, etc.)
\def\Dedication{%
This thesis is affectionally dedicated to my loving wife Alena and amazing
daughter Anna.
I would also like to thank my supervisor, professor Josef Štěpánek for his
patience, mentorship, and help with the thesis.
Finally, I would like to thank the members of the Division of Biomolecular
Physics for the warm, inspiring, and helpful environment.
}

% Abstract (recommended length around 80-200 words; this is not a copy of your
% thesis assignment!)
\def\Abstract{%
Despite the decades of intensive research, nucleic acids (NA) represent still a
permanent object of structural studies.
Within the framework of the doctoral work, the apparatus for measurement of UV
excited resonance Raman spectra (UV RRS) was built up and optimized.
A realistic and complex interpretation table was prepared based on analysis of
published data and extensive series of UV RRS measurements on NA model
structures, mononucleotides, and polynucleotides.
The established methodology was verified when applied in several structural
studies of nucleic acids, mainly the study of the influence of magnesium ions
on the equilibrium between duplexes and triplexes formed by PolyA and PolyU
homopolynucleotides, a study of temperature-induced structural changes in
DNA double helix and DNA hairpin, and investigation of slow structural
transitions of guanine quadruplexes induced by the presence of potassium ions.
The results of the test measurements and the above-mentioned studies have shown
that the created methodology for studying UV RRS of nucleic acids brings most
of the expected benefits of the resonance excitation: the possibility of Raman
scattering measurements at the same concentrations as in the case of UV
absorption, high sensitivity to fine temperature-induced structural changes
and good interpretability of the spectra obtained.
}

% 3 to 5 keywords (recommended), each enclosed in curly braces
\def\Keywords{%
{nucleic acids}, {UV RRS}, {PCA}, {UV Raman spectrometer}, {RRS markers}
}
