\chapter*{Conclusions}
\addcontentsline{toc}{chapter}{Conclusions}

In the framework of the doctoral work, the apparatus for measurement of UV
excited resonance Raman (UV RRS) spectra was built up and optimized.
The backscattering configuration with a sample located in a classic quartz
cuvette with intensive mixing was found as the most suitable for nucleic acids
(NA) investigation.
A precision temperature control was set up for this arrangement.
Furthermore, optimal parameters for measuring the NA spectra, relating to the
sample concentration, excitation power, and a time regime of the signal
accumulation were determined.
A precision spectral calibration system and semi-automatic data treatment
process were created, the latter consisting of erasing the cosmic radiation
spikes, intensity normalization, subtraction of the spectra of the solvent and
the cell wall, and the base line straightening.

In addition, a realistic and complex interpretation table was prepared, based
on analysis of published data and extensive series of UV RRS measurements on NA
model structures, mononucleotides and polynucleotides.
The table has proved to be a very necessary basis for analyzing UV RRS spectra
of NA segments.

The established methodology was very successfully verified when applied in
several structural studies of nucleic acids.
The most important ones are described in the dissertation.
The first of them concerned the influence of magnesium ions on the equilibrium
between duplexes and triplexes formed by PolyA and PolyU homopolynucleotides.
Furthermore, it was the detection of temperature-induced structural changes in
DNA double helix and DNA hairpin in the temperature region of their melting as
well as at lower and higher temperatures.
Finally, it was monitoring of slow structural transitions of guanine
quadruplexes induced by the presence of potassium ions.

The results of the test measurements and the above-mentioned studies have shown
that the created methodology for studying UV RRS of nucleic acids brings most
of the expected benefits of the resonance excitation: the possibility of Raman
scattering measurements at the same concentrations as in the case of UV
absorption, high sensitivity to fine temperature-induced structural changes and
good interpretability of the spectra obtained.
